\documentclass{abntpuc}

\usepackage[pdftex]{graphicx}

\usepackage{fancyhdr}

\begin{document}

% --- Campos que formarao: capa, folha de rosto, folha de aprovacao  ---

\curso{Bacharelado em Sistemas de Informação}

\autor{Alexandre Gonzaga Mendes}

\titulo{DEFINIR O NOME} % Caixa alta

\subtitulo{Subtítulo do Trabalho} % Caixa baixa

\cidade{Belo Horizonte}

\ano{2013}

\trabalho{Monografia} % Tipo de trabalho: Monografia, Dissertacao, Tese...

\grau{Bacharel em Sistemas de Informação} % Grau do trabalho: Bacharel em..., Mestre em..., Doutor em...

\orientador{João Caram Adriano}

\avaliadorA{Nome do Avaliador 1}

\avaliadorB{Nome do Avaliador 2}

\datacompleta{DD de MM de 2013} % Dia, mes e ano

% --- CAPA ---

\capa

% --- FOLHA DE ROSTO ---

\folharosto

% --- FOLHA DE APROVACAO ---

\folhaaprovacao

% --- DEDICATORIA (elemento opcional) ---

\dedicatoria
{
\textit{Dedicatória: Página onde o autor presta homenagem a uma ou mais pessoas.
O layout desta página fica a critério do autor, mas o tipo e tamanho de letras são definidos pela ABNT.}
}

% --- AGRADECIMENTOS (elemento opcional) ---

\agradecimentos
{
Agradecimentos a pessoas que contribuíram para o desenvolvimento do trabalho.
Agradecimentos a pessoas que contribuíram para o desenvolvimento do trabalho.
}

% --- EPIGRAFE (elemento opcional) ---

\epigrafe
{
\textit{Epígrafe: Pensamentos retirados de um livro, uma música, um poema, normalmente relacionados ao tema do trabalho.
Deve ser elaborada conforme norma NBR 10520/2002. Apresentação de citações em documentos.
Se desejar, a epígrafe pode ser grafada em itálico.
Ao final do trabalho deve-se fazer a referência completa da publicação de onde a epígrafe foi retirada.}
}

% --- RESUMO ---

\resumo
{
Apresentação concisa dos pontos relevantes do texto. Deve ressaltar o objetivo, o método, resultados e conclusões do trabalho. Deve-se utilizar o verbo na voz ativa ou terceira pessoa do singular. O resumo não deve conter citações ou indicações bibliográficas.
} % Resumo
{
Ao final do resumo deve-se elaborar palavras-chave representativas do conteúdo do trabalho, separadas entre si por um ponto.
} % Palavras-chave

% --- ABSTRACT ---

\abstract
{
Versão do resumo em idioma de divulgação internacional. Deve ser a tradução literal do resumo em português e apresentar palavras- chave no mesmo idioma, logo abaixo do texto, separadas entre si por um ponto.
} % Abstract
{
O resumo em língua estrangeira também deve conter palavras-chave representativas do conteúdo do trabalho, separadas entre si por um ponto.
} % Keywords

% --- LISTA DE FIGURAS, LISTA DE TABELAS, SUMARIO ---

\listafiguras
\listatabelas
\listasiglas {
\sigla{S1}{Sigla 1}
\sigla{S2}{Sigla 2}
\sigla{S3}{Sigla 3}
}
\sumario

% --- TEXTO ---
\capitulo{INTRODUÇÃO}

\iniciocapitulo

O trabalho consiste na distribuição das disciplinas dos diversos cursos de graduação e pós-graduação apresentados pelos colegiados, em salas, atraves do sistema que será criado todo inicio de semestre assim que todos os colegiados já tenham enviado suas necessidades para aquele semestres.\par

Para realização da alocação, faz-se necessário o conhecimento das salas existentes para atender a demanda, alem de conhecer sua capacidade e suas características. A partir deste passo, faz-se necessário o levantamento das solicitações enviadas pelos colegiados, para alocar as disciplinas em salas adequadas de forma a atender todas as solicitações de forma otimizada.\par

O principal problema atualmente é a falta de salas adequadas para a distribuição das disciplinas, uma vez que as turmas atuais estão com um numero de alunos matriculados acima da capacidade das salas, devido a este problema o sistema deve propor um relatório de disciplinas que não atenderam a capacidade do prédio para que o responsável pela alocação possa negociar em prédios de outros cursos a alocação das disciplinas que não poderão ser alocadas no prédio que o sistema executou a alocação.\par

\secao{Justificativas e Objetivos}

	Este trabalho envolve conhecimentos de analise de sistemas e desenvolvimento almejando um sistema capaz tratar e otimizar a execução do problema de alocação de salas de uma universidade. Este trabalho possui um grande valor uma vez que pode facilita a vida do responsável pela alocação das salas, por se tratar de um trabalho manual, trabalhoso e que para a execução são necessárias em torno de trinta horas que poderiam estar sendo utilizadas para uma tarefa mais importante.\par

Com este trabalho objetiva-se:\par

	- Desenvolvimento do sistema.\par
	- Apresentação de um modelo de algorítimo que atenda o problema proposto.\par
	- Otimizar o tempo do gestor.\par
	- Eficiência na geração dos relatórios.\par

\secao{Organização do Trabalho}

Este trabalho está definido da seguinte forma, foi dividido em cinco capítulos, sendo este capítulo 1 e mais quatro outros.\par

O capítulo 2 apresenta o referencial teórico do trabalho, descrevendo os conceitos utilizados para o desenvolvimento do projeto proposto: conceitos de ---------------------------------------\par

No capítulo 3 é apresentada a metodologia do sistema e as tecnologias adotadas para desenvolvimento da solução.\par

No capítulo 4 iremos descrever e citar detalhadamente as características e propostas de desenvolvimento do sistema -------, proposto para este trabalho.\par

A conclusão deste trabalho e planos futuros são mostrados no Capítulo 5.\par
\capitulo{REFERENCIAL TEÓRICO}

\secao{Problemas de Otimização}

Otimização é o processo de encontrar a melhor solução, também chamada de solução ótima para um determinado problema \cite{timoteo2005desenvolvimento}.\par

De acordo com \cite{steiglitz1982combinatorial} a constituição de um problema de otimização se deve aos termos vizinhança, ótimo local e ótimo global. O termo vizinhança se trata de um subconjunto do conjunto do problema, ótimo local pode ser tratado como o melhor resultado em uma vizinhança, e ótimo global é a melhor solução encontrada no conjunto de acordo com a função objetivo.\par

Conforme cita \cite{raupp2003introduccao}, o problema de otimização combinatória pode ser denominado como a ação de maximizar ou minimizar uma função objetiva de diversas variáveis sujeita a um conjunto de restrições, dentro de contexto.\par

De acordo com a figura 1 pode se observar a relação entre ótimo local e ótimo global em um problema típico de minimização.

\begin{figure}[!htb]
\caption[Representação de um problema de minimização com ótimos locais]{Representação de um problema de minimização com ótimos locais}
\label{fig:figura2}
\centering
\includegraphics[scale=0.55]{imagens/problemaOtimizacao.png}
\\ \textbf{\footnotesize Fonte: \cite{timoteo2005desenvolvimento}}
\end{figure}

Segundo \cite{steiglitz1982combinatorial}, os problemas de otimização são divididos em duas categorias, problemas com variáveis contínuas e problemas com variáveis discretas. Problemas com variáveis discretas também podem ser conhecidos como Problemas de Otimização Combinatória (POC).\par

De acordo com \cite{opac-b1092847} problemas do tipo POC tratam do estudo matemático para encontrar agrupamentos, arranjos ou a seleção ótima de objetos discretos, logo, não permitindo, a utilização de métodos clássicos de otimização contínua para sua resolução.\par


Segundo \cite{golbarg2000otimizaccao} a ocorrência de problemas de otimização combinatória podem acontecer em diversas áreas, projetos de sistemas de distribuição de energia elétrica, posicionamento de satélites, roteamento ou escalonamento de veículos, sequenciamento de genes e DNA, classificação de plantas e animais.\par

De acordo com \cite{deleonardo} em problemas de otimização combinatória, cujo universo de dados é grande e existe um grande número de combinações, o que torna inviável a análise de todas soluções possíveis em um tempo adequado, utilizamos as heurísticas, também conhecidas como algoritmos heurísticos, que são métodos que compõem uma gama de soluções para problemas de otimização combinatória.

\secao{Heurística}

O termo heurística é derivado do grego \textit{heuriskein}, o que significa descobrir ou achar. De acordo com \cite{timoteo2005desenvolvimento} o significado da palavra em pesquisa operacional vai um pouco além da raiz etimológica. Segundo \cite{steiglitz1982combinatorial}, heurísticas são consideradas métodos de aproximação ou métodos de busca de solução, deve se levar em consideração que não exista uma garantia formal de seu desempenho e uma garantia de estas heurísticas que iram encontrar uma solução. As heurísticas, apesar de não garantirem encontrar a solução ótima para um problema, procuram por soluções consideradas de boa qualidade em um tempo computacional razoável.\par

Segundo \cite{evans1992optimization} heurísticas são necessárias para implementação de problemas NP Difícil, caso deseje-se resolver tais problemas em um tempo  razoável.\par

Ressalta-se que dentre as heurísticas, as chamadas meta-heurísticas, merecem especial atenção pois adotam técnicas para amenizar, a dificuldade que os métodos heurísticos têm de escapar dos ótimos locais. As meta-heurísticas podem partir em busca de regiões mais promissoras no espaço de soluções, alem disto, as meta-heurísticas possuem grande abrangência, podendo ser aplicada à maioria dos problemas de otimização combinatória.\cite{nascimento2005aplicaccao}\par

Segundo \cite{adrianocesar} uma heurística é a instanciação de uma meta-heurística, ou seja, a aplicação da mesma em um problema específico de otimização.\par

Como exemplos de meta-heurísticas temos Busca Tabu (\textit{Tabu Search}), Otimização por Colônias de Formigas (\textit{Ant Colony Optimization}), Recozimento Simulado (\textit{Simulated Annealing}) e Algoritmo Genético (\textit{Genetic Algorithm}).\par

\subsecao{Busca Tabu}

A metaheurística BT foi inicialmente desenvolvida por \cite{glover1986future} como uma proposta de
solução para problemas de programação inteira. A partir de então, o autor formalizou esta
técnica e publicou uma série de trabalhos contendo diversas aplicações da mesma. A
metaheurística BT utiliza uma lista contendo o histórico da evolução do processo de busca, de
modo a evitar ciclagem; incorpora uma estratégia de balanceamento entre os movimentos
aceitos, rejeitados e aspirados; e adota procedimentos de diversificação e intensificação para o
processo de busca.\par
\cite{souza2000} e \cite{white2004using} ressaltam a existência de um mecanismo, relacionado com
a Lista Tabu, que anula o status tabu de um movimento, denominado função de aspiração. Se
um movimento pode proporcionar uma melhora considerável da função objetivo, então o
status tabu é abandonado e a solução resultante é aceita como potencial vizinho.

\subsecao{Algoritmo da colônia de formigas}	

	Achar uma referencia sobre o algoritimo da colonia

\subsecao{Recozimento Simulado}

Técnica de busca local probabilística, proposta originalmente por \cite{kirkpatrick1983optimization}, que se fundamenta em uma analogia com a termodinâmica, ao simular o 
resfriamento de um conjunto de átomos aquecidos. Isto é, conforme \cite{noronha2003abordagem} em 
analogia a física da matéria: levando um cristal a sua temperatura de fusão, as moléculas 
estão desordenadas e se agitam livremente. Ao resfriar-se a amostra de maneira 
infinitamente lenta, as moléculas vão adquirir a estrutura cristalina estável que tem um 
nível de energia mais baixo possível. 
Conforme \cite{aarts1988simulated} a analogia com a otimização (combinatória ou não) 
é bastante direta. Os estados da matéria são as soluções realizáveis, a quantidade objetiva 
substitui a energia, os estados metaestáveis da matéria sendo ótimos locais e a estrutura 
cristalina corresponde ao ótimo global. 
Segundo \cite{reeves1993modern}, a temperatura Tassume, inicialmente, um valor 
elevado T0e o procedimento pára quando a temperatura chega a um valor próximo de zero 
e nenhuma solução que piore o valor da função objetivo é mais aceita, isto é, quando o 
sistema está estável.\par
Mais informações em \cite{reeves1993modern} e \cite{kirkpatrick1983optimization}. 

\subsecao{Algoritmos genéticos}

Conforme cita \cite{oliveira2005algoritmo}, os algoritmos genéticos foram introduzidos 
por \cite{holland1975adaptation}, com intuito de aplicar a teoria da evolução das espécies 
elaborada por \cite{darwin1968origin} utilizando os conceitos da evolução biológica como 
genes, cromossomos, cruzamento, mutação e seleção na computação procurando explicar rigorosamente processos de adaptação em sistemas naturais e desenvolver sistemas artificiais (simulados em computador) que mantenham os mecanismos originais, encontrados em sistemas naturais.\par

Segundo \cite{oliveira2005algoritmo}, o processo de evolução executado por um algoritmo genético corresponde a um procedimento de busca no espaço de soluções potenciais para o problema e, como enfatiza \cite{michalewicz1996evolutionary}, esta busca requer um equilíbrio entre dois objetivos aparentemente conflitantes: a procura das melhores soluções na região que se apresenta promissora ou fase de intensificação e a procura de outra região ou exploração do espaço de busca, também conhecida como diversificação.\par
Ainda segundo \cite{oliveira2005algoritmo}, os algoritmos genéticos têm se mostrado ferramentas poderosas para resolver problemas onde o espaço de busca é muito grande e os métodos convencionais se mostraram ineficientes.\par

Mitchel \cite{mitchell1998introduction} cita que a terminologia biológica é muito importante para a compreensão do funcionamento dos algoritmos genéticos. Eis os principais termos: \par
•  Cromossomo: estrutura que representa uma determinada característica da solução ou a própria solução; \par
•  Gene: característica particular de um cromossomo. O cromossomo é composto por um ou mais genes. \par
•  Alelo: valor de determinado gene;\par
•  Locus: determinada posição do gene no cromossomo;\par
•  Genótipo: estrutura que codifica uma solução. Um genótipo pode ser formado por um ou mais cromossomos; \par
•  Fenótipo: decodificação ou o significado da estrutura; \par
•  Fitness: significa aptidão. O quanto o indivíduo é apto para determinado ambiente; \par
As principais características que diferenciam os algoritmos genéticos de métodos tradicionais são \cite{goldberg1989genetic}: \par
•  Parâmetros: os algoritmos genéticos trabalham com a codificação dos parâmetros e não com os parâmetros propriamente; \par
•  Número de soluções: os algoritmos genéticos trabalham com uma população de indivíduos (representando um conjunto de soluções) e não com uma única solução; \par

•  Avaliação das soluções: os algoritmos genéticos utilizam informações de custo ou recompensa penalizando ou premiando determinadas características das soluções; \par

•  Regras: os algoritmos genéticos utilizam regras probabilísticas e não determinísticas; \par

O algoritmo genético é uma forma da estratégia gerar-e-testar realizando os testes baseados nos parâmetros da evolução biológica. Uma desvantagem notável é a variação dos operadores genéticos do algoritmo em cada problema. Dessa forma, para resolução de determinado problema, torna-se necessário um estudo particular a respeito do mesmo. \par

O algoritmo genético atua sobre uma população fazendo com que esta evolua de acordo com uma função de avaliação. O funcionamento é iterativo iniciando com a geração de uma população inicial que pode ser aleatória ou não, seguida do processo de avaliação, seleção, cruzamento e mutação, que ocorre a cada iteração até que seja atingido algum critério de parada. Os passos gerais de um algoritmo genético são ilustrados na figura 
Figura XXXX. Cada passo pode ser realizado de várias maneiras e pode variar de problema para problema \cite{timoteo2005desenvolvimento}.\par 

Figura Etapas de um Algoritmo Genético Básico 

\secao{TimeTabling}

Segundo \cite{kripkasimulated} os problemas de programação de horários (PPH), também conhecidos como \textit{TimeTabling}, são os problemas que mais se destacam nas organizações acadêmicas. De acrodo com \cite{schaerf1999survey} estes problemas são divididos em três categorias \textit{school timetabling}, \textit{course timetabling} e \textit{examination timetabling}.\par

\textit{School TimeTabling}: Se trata basicamente da geração de horários semanais, em escolas de segundo grau, onde deve-se evitar os choques entre os horários das disciplinas e que cada professor receba apenas uma turma para cada horário. Neste caso o aluno recebe um número fixo de disciplinas a serem cursadas.\par

\textit{Course TimeTabling}: Diz respeito à alocação de aulas de uma universidade típica. Neste problema os alunos podem escolher as matérias em que vão se matricular, portando o problema tem como objetivo minimizar os possíveis choques entre as disciplinas, professores e horários disponibilizados pela instituição de ensino.\par

\textit{Examination TimeTabling}: Aborda o problema de programação de horários dos exames da instituição, de maneira que, disciplinas que tenham alunos em comum, distanciem o máximo possível as datas dos exames.\par

Segundo \cite{pinheiro2001ambiente} o problema de programação de horários vem sendo abordado desde a década de 60, sendo que os primeiros trabalhos a se destacarem foram realizados na década de 80.\par

O Problema de Alocação de Salas (PAS) também conhecido como \textit{Classroom Assignment} é tratado como parte do problema de programação de cursos universitários \textit{course timetabling}. Segundo \cite{marinho2005heuristicas} varias instituições universitárias se deparam com o PAS durante o início de cada semestre letivo, este problema é considerado NP-Difícil por \cite{even1975complexity} e \cite{carter1992classroom}, com isto, a determinação da solução ótima do problema, em um período de tempo aceitável se torna uma tarefa difícil.\par
Segundo \cite{kripkasimulated} o problema deve considerar que as disciplinas dos cursos universitários já tenham seus horários de início e de término definidos. O problema se resume então na alocação das disciplinas às salas desta universidade respeitando os horários destas disciplinas e as demais restrições exigidas.

Segundo \cite{souza2000} boa parte das universidades ainda resolvem este problema de forma manual, o que torna o processo árduo e demorado, podendo levar vários dias para ser concluído.\par

Uma vez que é de extrema dificuldade encontrar a solução ótima do PAS em tempo razoável, este problema é normalmente tratado através de técnicas heurísticas, que apesar de não garantirem encontrar a solução ótima do problema, são capazes de retornar uma solução de qualidade em um tempo adequado. As meta-heurísticas surgiram como uma alternativa para amenizar a dificuldade que os métodos heurísticos tem de escapar dos chamados ótimos locais.\cite{nascimento2005aplicaccao}.\par

Segundo \cite{even1975complexity} o PAS pertence a classe de Problemas de Otimização Combinatória (POC).\par


\secao{Trabalhos Relacionados}
trabalho do marinho que usa tabu.
trabalho da silvia que usa Recozimento Simulado (Simulated Annealing).
trabalho da leonardo que usa Algoritmos Genéticos (AG).
achar algum trabalho que utiliza o algoritimo das formigas.
falar porque o trabalho do cara se assemelha ao meu trabalho.
% Nome do capitulo
\chapter{Metodologia}
% Label para referenciar
\label{cap:metodologia}
% Diminuir espacamento entre titulo e texto
\vspace{-1.9cm}

% Texto do capitulo
Primeiramente para esse projeto foram realizadas as pesquisas bibliogr�ficas. Trabalhos de outros autores que de alguma forma se relacionam com a monografia foram estudados com a finalidade fornecer uma base de conhecimento sobre o tema, como por exemplo,  o entendimento de todo o processo de descoberta do conhecimento e suas t�cnicas de minera��o de dados. Com isso foi poss�vel montar os cap�tulos de revis�o bibliogr�ficas abordando os principais conceitos utilizados durante o desenvolvimento.

No segundo momento, o desenvolvimento trata-se da aplica��o dos passos do processo de KDD sobre os Microdados do Censo da Educa��o Superior. Foram executados os seguintes passos: sele��o de dados, pr�-processamento, transforma��o, minera��o de dados e interpreta��o.Na sele��o de dados foi definida a fonte dos dados. No pr�-processamento removemos principalmente erros relacionados com a falta de informa��o. Na transforma��o os dados foram enriquecidos. Partindo para a minera��o de dados, os algoritmos de influencias, previs�o, clusteriza��o, e an�lise de cesta de compras foram aplicados. Finalmente, na interpreta��o dos dados os gr�ficos e an�lises encontradas s�o apresentados.

\capitulo{SISTEMA DESENVOLVIDO}

\iniciocapitulo
O capítulo descreve o sistema desenvolvido neste trabalho, tem como objetivo o sistema, otimizar a alocação das disciplinas envidas pelos colegiados em salas de uma universidade, neste caso as salas do prédio FAFICH na UFMG. Por se tratar de um problema especifico do local torna-se difícil o trabalho de encontrar tecnologias disponíveis para otimização do problema, neste caso o desenvolvimento de um sistema que atenda todas as necessidades é de grande importância para o responsável pela alocação.\par

\secao{Modelagem}

\subsecao{Diagramas de caso de uso}

A Figura XX descreve todas as funcionalidades que o sistema possui, essas funcionalidades foram dividias em 2 atores "Gerente" e "Sistema" cada um ligado com suas respectivas funcionalidades, porem, o "Gerente" pode acessar o ator "Sistema" para ter acessos funcionalidades que são encontradas no mesmo. O sentido das setas informa o que cada ator pode acessar no sistema.\par

As funcionalidades controle de turnos, controle de horários, controle de prédios, controle de salas, controle de cursos, controle de colegiados, controle de períodos e controle de disciplinas são disponibilizadas através de módulos CRUD para que o responsável tenha total controle sobre o que será alocado, onde e como.\par

Já as funcionalidades alocação de horários e geração de relatórios, são rotinas executadas pelo "Sistema". A funcionalidade alocação de horários é uma rotina que utiliza conceitos de algorítimo genético para encontrar a melhor solução do problema e a funcionalidade geração de relatórios mostra para o "Gerente" o melhor resultado de alocação encontrado pelo algorítimo.\par

\begin{figure}[!htb]
\caption[Diagrama de Caso de Uso]{Diagrama de Caso de Uso}
\label{fig:figura1}
\centering
\includegraphics[scale=0.4]{imagens/diagramaCasoUso.png}
\\ \textbf{\footnotesize Fonte: Desenvolvido pelo autor}
\end{figure}

\subsecao{Diagrama de Entidade Relacionamento}

Na Figura XX mostra o relacionamento das tabelas no sistema atraves do diagrama de entidade relacionamento.\par

As tabelas curso, colegiado, período, disciplina, sala, prédio, turno, horário são utilizadas para armazenar as informações dos respectivos objetos que são controlados pelo CRUD e suas respectivas funcionalidades.\par

A tabela relacionamento\_disciplina\_horário, salva as obrigatoriedades dos horários das disciplinas uma vez que todos os horários são montados pelos colegiados e não pelo responsável pela alocação das disciplinas nas salas.\par

A tabela alocação contem o relacionamento de disciplina, horário e sala o que corresponde a melhor alocação encontrada pelo algorítimo. O item disciplina pode ter o seu valor como nulo o que significa que em um especifico horário e em uma determinada sala não existe disciplina alocada.\par

A tabela parâmetro é utilizada apenas para guardar os últimos parâmetros utilizados na execução do algorítimo genético.\par

\begin{figure}[!htb]
\caption[Diagrama de Entidade Relacionamento]{Diagrama de Entidade Relacionamento}
\label{fig:figura2}
\centering
\includegraphics[scale=0.5]{imagens/diagramaEntidadeRelacionamento.png}
\\ \textbf{\footnotesize Fonte: Desenvolvido pelo autor}
\end{figure}

\subsecao{Diagramas de classe}

Este diagrama é das classes do algorítimo genético, as classes do sistema em geral não serão abordados. Quando terminar o codigo revisar e explicar.\par
Lorem ipsum dolor sit amet, consectetur adipiscing elit. Donec vestibulum mauris at velit varius aliquet. Nulla at risus vehicula, tempus orci sagittis, molestie nibh. Fusce sodales sollicitudin viverra. Aliquam erat volutpat. Nullam nec odio mi. Suspendisse vel mauris felis. In mi diam, auctor vel feugiat ut, pulvinar ac ipsum. In a convallis arcu. Integer a augue accue rutrum mauris. Phasellus id massa a lorem semper placerat eget eu urna.\par

\begin{figure}[!htb]
\caption[Diagrama de Classe]{Diagrama de Classe}
\label{fig:figura3}
\centering
\includegraphics[scale=0.6]{imagens/diagramaClasse.png}
\\ \textbf{\footnotesize Fonte: Desenvolvido pelo autor}
\end{figure}

\secao{Algorítimo Genético}

Após a execução da modelagem do sistema com pleno conhecimento do problema e o levantamento bibliográfico sobre algorítimos genéticos, foi feito o relacionamento do problema com os termos da biologia. Foram encontradas varias fontes que agregaram valor para o desenvolvimento do trabalho, porem algumas modificação foram feitas para que o modelo tratado por outros autores funcionasse adequadamente para a resolução do problema proposto para este trabalho. A seguir serão apresentados os itens da biologia utilizados no desenvolvimento do algorítimo juntamente com sua ligação com o problema.\par

\subsecao{Indivíduo}

Alguns trabalhos tratam os termos indivíduo e cromossomo como a mesma representação biológica, neste trabalho o termo indivíduo é composto pelo cromossomo juntamente com a pontuação adquirida após a execução do método de calculo de fitness, já o termo cromossomo se refere a combinação dos genes do indivíduo, a figura XX demonstra a representação de um indivíduo.\par

\begin{figure}[!htb]
\caption[Representação Individuo]{Representação Individuo}
\label{fig:figura7}
\centering
\includegraphics[scale=0.8]{imagens/representacaoIndividuo.png}
\\ \textbf{\footnotesize Fonte: Desenvolvido pelo autor}
\end{figure}

Um cromossomo é uma sequência de genes, neste trabalho esta sequência tem um tamanho fixo e pode ser medido pela seguinte formula (número de Salas * número de Horários * número de dias da semana). Inicialmente o individuo contem todos as variáveis relacionamentos de obrigatoriedade nulas, estas variáveis serão preenchidas randomicamente na criação da população inicial que em breve será explicada. O cromossomo do indivíduo preenchido com os relacionamentos de obrigatoriedades representa uma alocação, está alocação é medida pela sua pontuação de fitness, podendo ser ou não o resultado do problema. Os valores utilizados na representação do cromossomo na figura XX são os ID's do relacionamento de obrigatoriedade entre horário e disciplina, em caso de horário vago em um determinado gene esta variável terá o valor nulo.\par

\begin{figure}[!htb]
\caption[Representação Cromossomo]{Representação Cromossomo}
\label{fig:figura6}
\centering
\includegraphics[scale=0.9]{imagens/representacaoCromossomo.png}
\\ \textbf{\footnotesize Fonte: Desenvolvido pelo autor}
\end{figure}

O termo gene representado pela figura XX possui uma combinação de quatro variáveis sala, dia da semana, horário e o relacionamento de obrigatoriedade 
"disciplina horário". As variáveis sala, horário e dia da semana são fixas, não podem ser nulas uma vez que todas as combinações possíveis destas três variáveis formam um cromossomo que tem um tamanho fixo para todos os indivíduos. \par

Um gene com o relacionamento "disciplina horário" igual a nulo, representa um horário vago para aquela combinação especifica de sala, horário e dia da semana.\par

\begin{figure}[!htb]
\caption[Representação Gene]{Representação Gene}
\label{fig:figura5}
\centering
\includegraphics[scale=0.7]{imagens/representacaoGene.png}
\\ \textbf{\footnotesize Fonte: Desenvolvido pelo autor}
\end{figure}

O fitness do indivíduo é calculo através da função objetivo que será abordada em outro tópico brevemente explicando se trada de um valor que vai de 0 a 100 onde 100 é a pontuação máxima de do indivíduo está nota é alcançada quando todos os requisitos de alocação desejados foram atendidos.

\subsecao{População}

Uma população, quando relacionada aos termos genéticos se trata de um conjunto de indivíduos, também representa uma interação do algorítimo genético, ou seja uma geração.A manipulação da população e de suas propriedades é feita através de parâmetros enviados antes da execução do algorítimo, estes parâmetros são elitismo, taxa de crossover, taxa de mutação, tamanho da população e número máximo de gerações.

De acordo com os parâmetros passados é criada a população inicial. Para cada indivíduo criado é utilizado um método para inserir randomicamente todos os registros de relacionamento de obrigatoriedade entre disciplina e horário em cada um dos genes que previamente foram criados como nulo, estes indivíduos são criados até a população atinga o tamanho da população que deve ser igual ao parâmetro tamanho da população.\par

Para cada geração é criada uma nova população a partir da população criada na geração anterior, se o operador genético elitismo estiver como valor verdadeiro, iniciamos está nova população com 20\% dos melhores indivíduos da população anterior, os melhores indivíduos de uma População são indicados pelas maiores pontuações de fitness de da população anterior.\par

Durante a criação da nova população podemos ter duas operações ocorrendo mutação e crossover, para a execução destes operadores genéticos são utilizadas porcentagens enviadas pelos parâmetros do algorítimo. Para que estas operações genéticas aconteçam são utilizados valores randomicos para serem comparados com as taxas de mutação e crossover. Em cada interação da criação desta nova população são selecionados por torneio os pais que serão utilizados no crossover, se a condição da taxa de execução for verdadeira os pais serão descartados e os filhos gerados a partir dos genes dos pais, serão adicionados na nova população, em caso de falso os pais serão adicionados na nova população.\par
Para se utilizar a operação de mutação novamente é utilizado outro valor randominco se a condição for verdadeira um indivíduo da população anterior é selecionado e sofrerá a mutação genética, o individuo antes da mutação genética é descartado e o novo indivíduo que sofreu a mutação é adicionado na nova população. O fluxo de uma nova população é descrito na figura XX.\par

\begin{figure}[!htb]
\caption[Fluxo Nova População]{Fluxo Nova População}
\label{fig:figura8}
\centering
\includegraphics[scale=0.7]{imagens/fluxoNovaPopulacao.png}
\\ \textbf{\footnotesize Fonte: Desenvolvido pelo autor}
\end{figure}

\subsecao{Operadores genéticos}

Este trabalho utiliza elitismo como operador genético ao se inciar uma nova população, vinte porcento dos melhores indivíduos são escolhidos para compor a nova população.O elitismo é calculdo atravez da função objetiva criada para o problema especifico do trabalho.\par

Para selecionar os inviduos para realizar o crossover é utilizado metodo de seleção por torneio, são escolhidos três Individuos da população anterior, e destes três são escolhidos os dois com maior pountuação de fitness, os dois Individuos selecionados são enviados para o crossover.\par

Mutacao é a inversão genetica dos genes de um Individuo escolhido randomicamente da população anterior, apos a realização da mutação genetica o individuo é inserido na nova população.\par

Primeiramente são escolhidos dois genes do cromossomo, apos a escolha randomica dos itens a serem trocados, é feita a troca dos genes e retornado um ndividuo que tem a composição genetica apos a alteração, Apos a mutação este Individuo recebe uma nova nota de fitness de acordo com a sua nova sequencia de Genes e sua adaptação no ambiente, está nota pode ser maior ou menor do que a anterior.\par

\begin{figure}[!htb]
\caption[Representação Mutação]{Representação Mutação}
\label{fig:figura8}
\centering
\includegraphics[scale=0.7]{imagens/representacaoMutacao.png}
\\ \textbf{\footnotesize Fonte: Desenvolvido pelo autor}
\end{figure}

Crossover é o cruzamento dos individuos selecionados pela seleção torneio. Explicar com figuras. apos o cruzamento dos individuo os dois filhos gerados são inseridos na nova população.\par

\subsecao{Definição da função objetivo}

Somatorio disso

Para o calculo do fitness foram definidos pesos para modelagem da função, estes pesos podem ser configurados de acordo com a necessidade da alocação.

Graduação alocada ganha 05 de peso

Pós graduação alocada ganha 03 de peso

Periodos na mesma sala cada um ganha 05 de peso * o numero do periodo

Quanditadade de vagas igual a da sala 05 de peso

não optativa ganha 5

optativa ganha 3

iliminacao atendida 5

Criar a função matematica com as legendas conforme o trabalho 117.pdf

Falar o numero de salas, o numero de horarios, o numero de curos o numero de colegiados o numero de periodos o numero de disciplinas para cada colegiado.......

restrições 


falar um pouco das restrições e enumeralas

As disciplinas não podem ser alocadas em horarios direfentes dos que já foram pré definidos pelo colegiado.

As discplinas devem ter apenas a quantidade de alocações necessarias.

As disciplinas devem respeitar a capacidade da sala.

As diciplinas não optativas tem preferencia de alcação na mesma sala.

Preferencias por salas claras ou escuras

Restrição 1 


Fitness

Para se iniciar o calculo do fitness são verificados todos os horarios já alocados somando os pesos se adequados.

para cada gene

se tem horario alocado 

horario bate

capacidade da turma

turma graduacao

optativa

iluminacao

soma tudo

fim se tem alocação

soma tudo

fim para cada gene

Calculo do fitness01 somatatoriox100/colocar algum valor  para dividir não sei ainda

calculo fitness02 penaliza disciplinas com mais alocação do que se deve

para cada gene 

para cada disciplina 

soma

fim

Calculo fitness02 -= fitnes01 x (1 - (total alocados - total necessario/ dividir pelo numero possivel de alocações))
fim

calculo fitness03 penalidade por capacidade

para cada gente

se a sala tiver capacidade diferente

fim

calculo fitness03 = fitness02 x (1 - (numero de erros /  numero de possiveis alocações )))


calculo fitness04 preferencias clara ou escura

para cada gene 

se tiver com o optativo errado 

fim	

calculo fitness04 = fitness03 x (1 - (numero de erros /  numero de possiveis alocações )))


O fitness04 é o resultado final

\subsecao{Fluxo do algoritimo}

parametros Algoritimo

O fluxo do algorimo conforme a imagem XX é iniciado pela criação da população inicial, para cada interação do algoritimo é verificado se a população contem o resultado e se o algoritimo não atingiu o numero de gerações pré defindas. Se as duas condições forem falsas o algoritimo cria uma nova População de acordo coma figura XX

\begin{figure}[!htb]
\caption[Fluxo Algoritimo]{Fluxo Algoritimo}
\label{fig:figura8}
\centering
\includegraphics[scale=0.7]{imagens/fluxoAlgoritimo.png}
\\ \textbf{\footnotesize Fonte: Desenvolvido pelo autor}
\end{figure}
\capitulo{RESULTADOS OBTIDOS}
	Despois do sistema implementado		

	Entrada processamento e saida

	ajuste do algoritomo de alocação

\capitulo{CONSIDERAÇÕES FINAIS}

\iniciocapitulo

*Relembra ro problema e comprar como o resultadoo bjetido.

Discussão dos resultados obtidos na pesquisa, onde se verificam as observações pessoais do autor. Poderá também apresentar sugestões de novas linhas de estudo. A conclusão deve estar de acordo com os objetivos do trabalho. A conclusão não deve apresentar citações ou interpretações de outros autores.

\secao{Trabalhos futuros}

Criar um DW para geração dos relatorios de acordo com a dimensão escolhida.

Utilização de outros algoritimos para a resolução do problema ex. algoritimos evolutivos formiga entre outros.

Pegar o feed back do usuario para melhoria na interface, e do algoritimo.

\capitulo{RESULTADOS OBTIDOS}


\iniciocapitulo

Após o sistema implementado foram realizadas uma bateria de testes para que os valores paramêtros inicias fossem definidos, o dados utiliados tem  em sua composição o total de 10 salas e um curso com o total de 46 disciplinas e duas respectivas obrigatoriedades, serão utilizados como paramêtros iniciais os dados representados na Figura~\ref{fig:parametrosAg}, no caso uma população com 140 individuos, 60\% de croosover 20\% de mutação, para entender melhor o andamento do algorítimo inicialmente será utilizada 400 gerações, o elitismos acontece em todos os testes realizados. Estes dados foram tirados como base na conclusão do trabalho de \cite{deleonardo}. Os testes realizados tiveram uma maquina com a seguinte configuração, Intel(R) Core(TM) i5 3.40GHz com 16Gb de mémoria RAM sob o sistema operacional Windows 7 64bit.\par

\begin{figure}[!htb]
\caption[Parâmetros utilizados para o teste inicial]{Parâmetros utilizados para o teste inicial}
\label{fig:parametrosAg}
\centering
\includegraphics[scale=0.5]{imagens/parametrosDoAlgoritimo.png}
\\ \textbf{\footnotesize Fonte: Desenvolvido pelo autor}
\end{figure}

Ao realizar o primeiro teste, que é realizado com o uso de  todos os parametros iniciais foi indentificado que o algoritímo fica preso em planicies, ou seja fica preso por muito tempo em uma solução sem evolução da mesma. Neste caso podemos notar no gráfico representado pela Figura X que o algorítmo no começo tem uma evolução rápida, mas em um determinado ponto ele perde seu poder de evolução e começa a caminha devagar em soluções que tem como valor da fuñção objetiva em torno de 50\% o que não faz da solução, uma solução que atenda ao objetivo deste trabalho.


Tentando identificar as possíveis causas desta aceleração de evolução do algorítmo foram realizados teste onde são retirados alguns parâmetros para identificar o que está acontecendo. Primeiramente foi realizado o teste onde o parâmetro de mutação não é utilizado. A figura X mostra o desepenho do algoritmo nesta situação. Podemos observar que em determinado ponto o algorítimo evolui rapidamente, porem o mesmo entra em uma planicie e não consegue mais prosseguir com sua evolução, isso se deve ao fato de um super-indivíduo ter dominado a população passando a caracteristicas do seus gens para todos os indivíduos da população. Isto acontece pois não ocorre a mutação que é responsável pela troca de genes aleatoriamente de um indivíduo.

Logo em seguida foi realizado o teste onde o parâmetro de crossover não é utilizado. A figura X representa a evolução do algoritmo de acordo com a velocidade com que acontece a evolução de acordo com as mutações podemos observar que, a evolução através da mutação é bem lenta e o algoritmo encontra varias planicies durante a evolução tornando assim o processo para encontrar a solução bem demorado.

Comparando os gráficos três gráficos gerados apresentados na figura X podemos observar que o crossover é responsável por acelerar a evolução e a mutação faz com que o algoritmo tenha uma nova diversidade de invidivíduos para que não ocorra a parada do mesmo em planicies.


Para acelerar a evolução do algoritmo como solução foi implementada na mutação, a mutação com melhoria genética onde o algorítimo ao realizar este operador genético ao invés de apenas realizar uma troca genética, ele procura fazer a troca dos genes, enviando as características de um gene para um lugar mais adequado que respeite a restrições do problema.

A figura X representa a comparação entre o primeiro primeiro gráfico gerado com o parâmetro de mutação com a nova solução gerada através da utilização da melhoria genética. Podemos observar uma grande melhoria na evolução do algorítmo.


Inicialmente o método de mutação com melhoria genética troca apenas um dos genes, foram então realizados testes trocando um gene, quatro genes e 10\% dos genes.O gráfico X mostra a comparação entre os três testes realizados. Podemos observar que quantos mais genes trocados durante a mutação maior é a evolução do algoritmo.

Após a escolha do numero de genes a serem trocados durante a melhoria genética, foi realizado um novo teste que é a utilização do operador genético de mutação com o de crossover o resultado é apresentado na Figura X. Falar o que aconteceu.


Logo em seguida foi aumentado o número de gerações para descobrir se o algoritmo encontraria a melhor solução e o mesmo encontro em X. O resultado pode ser encontrado no APENDICE X


Após está bateria de testes para descobrir sobre os passos dados pelo algoritmo foi realizado um teste com a carga completa que é XXXXXX.

Com o total de X horas e X gerações o algoritmo encontro a melhor solução que se encontra no APENDICE X.

\referencias{referencias}

%\usepackage[titletoc,toc,page]{appendix}
\renewcommand{\appendixtocname}{Ap\'endices}
\renewcommand{\appendixpagename}{Ap\'endices}



\end{document}