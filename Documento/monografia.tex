\documentclass{abntpuc}

\usepackage[pdftex]{graphicx}

\usepackage{fancyhdr}

\begin{document}

% --- Campos que formarao: capa, folha de rosto, folha de aprovacao  ---

\curso{Bacharelado em Sistemas de Informação}

\autor{Alexandre Gonzaga Mendes}

\titulo{DEFINIR O NOME} % Caixa alta

\subtitulo{Subtítulo do Trabalho} % Caixa baixa

\cidade{Belo Horizonte}

\ano{2013}

\trabalho{Monografia} % Tipo de trabalho: Monografia, Dissertacao, Tese...

\grau{Bacharel em Sistemas de Informação} % Grau do trabalho: Bacharel em..., Mestre em..., Doutor em...

\orientador{João Caram Adriano}

\avaliadorA{Nome do Avaliador 1}

\avaliadorB{Nome do Avaliador 2}

\datacompleta{DD de MM de 2013} % Dia, mes e ano

% --- CAPA ---

\capa

% --- FOLHA DE ROSTO ---

\folharosto

% --- FOLHA DE APROVACAO ---

\folhaaprovacao

% --- DEDICATORIA (elemento opcional) ---

\dedicatoria
{
\textit{Dedicatória: Página onde o autor presta homenagem a uma ou mais pessoas.
O layout desta página fica a critério do autor, mas o tipo e tamanho de letras são definidos pela ABNT.}
}

% --- AGRADECIMENTOS (elemento opcional) ---

\agradecimentos
{
Agradecimentos a pessoas que contribuíram para o desenvolvimento do trabalho.
Agradecimentos a pessoas que contribuíram para o desenvolvimento do trabalho.
}

% --- EPIGRAFE (elemento opcional) ---

\epigrafe
{
\textit{Epígrafe: Pensamentos retirados de um livro, uma música, um poema, normalmente relacionados ao tema do trabalho.
Deve ser elaborada conforme norma NBR 10520/2002. Apresentação de citações em documentos.
Se desejar, a epígrafe pode ser grafada em itálico.
Ao final do trabalho deve-se fazer a referência completa da publicação de onde a epígrafe foi retirada.}
}

% --- RESUMO ---

\resumo
{
Apresentação concisa dos pontos relevantes do texto. Deve ressaltar o objetivo, o método, resultados e conclusões do trabalho. Deve-se utilizar o verbo na voz ativa ou terceira pessoa do singular. O resumo não deve conter citações ou indicações bibliográficas.
} % Resumo
{
Ao final do resumo deve-se elaborar palavras-chave representativas do conteúdo do trabalho, separadas entre si por um ponto.
} % Palavras-chave

% --- ABSTRACT ---

\abstract
{
Versão do resumo em idioma de divulgação internacional. Deve ser a tradução literal do resumo em português e apresentar palavras- chave no mesmo idioma, logo abaixo do texto, separadas entre si por um ponto.
} % Abstract
{
O resumo em língua estrangeira também deve conter palavras-chave representativas do conteúdo do trabalho, separadas entre si por um ponto.
} % Keywords

% --- LISTA DE FIGURAS, LISTA DE TABELAS, SUMARIO ---

\listafiguras
\listatabelas
\listasiglas {
\sigla{S1}{Sigla 1}
\sigla{S2}{Sigla 2}
\sigla{S3}{Sigla 3}
}
\sumario

% --- TEXTO ---

\capitulo{INTRODUÇÃO}

\iniciocapitulo

O trabalho consiste na distribuição das disciplinas dos diversos cursos de graduação e pós-graduação apresentados pelos colegiados, em salas, atraves do sistema que será criado todo inicio de semestre assim que todos os colegiados já tenham enviado suas necessidades para aquele semestres.\par

Para realização da alocação, faz-se necessário o conhecimento das salas existentes para atender a demanda, alem de conhecer sua capacidade e suas características. A partir deste passo, faz-se necessário o levantamento das solicitações enviadas pelos colegiados, para alocar as disciplinas em salas adequadas de forma a atender todas as solicitações de forma otimizada.\par

O principal problema atualmente é a falta de salas adequadas para a distribuição das disciplinas, uma vez que as turmas atuais estão com um numero de alunos matriculados acima da capacidade das salas, devido a este problema o sistema deve propor um relatório de disciplinas que não atenderam a capacidade do prédio para que o responsável pela alocação possa negociar em prédios de outros cursos a alocação das disciplinas que não poderão ser alocadas no prédio que o sistema executou a alocação.\par

\secao{Justificativas e Objetivos}

	Este trabalho envolve conhecimentos de analise de sistemas e desenvolvimento almejando um sistema capaz tratar e otimizar a execução do problema de alocação de salas de uma universidade. Este trabalho possui um grande valor uma vez que pode facilita a vida do responsável pela alocação das salas, por se tratar de um trabalho manual, trabalhoso e que para a execução são necessárias em torno de trinta horas que poderiam estar sendo utilizadas para uma tarefa mais importante.\par

Com este trabalho objetiva-se:\par

	- Desenvolvimento do sistema.\par
	- Apresentação de um modelo de algorítimo que atenda o problema proposto.\par
	- Otimizar o tempo do gestor.\par
	- Eficiência na geração dos relatórios.\par

\secao{Organização do Trabalho}

Este trabalho está definido da seguinte forma, foi dividido em cinco capítulos, sendo este capítulo 1 e mais quatro outros.\par

O capítulo 2 apresenta o referencial teórico do trabalho, descrevendo os conceitos utilizados para o desenvolvimento do projeto proposto: conceitos de ---------------------------------------\par

No capítulo 3 é apresentada a metodologia do sistema e as tecnologias adotadas para desenvolvimento da solução.\par

No capítulo 4 iremos descrever e citar detalhadamente as características e propostas de desenvolvimento do sistema -------, proposto para este trabalho.\par

A conclusão deste trabalho e planos futuros são mostrados no Capítulo 5.\par

\capitulo{REVISÃO LITERARIA}

\secao{Descricão do problema}
Escrever sobre os objetivos e as justificativas

\secao{Complexidade do problema}
Este trabalho está definido da sequinte forma. Capitulo 1 Capitulo 2 Capitulo 3 Capitulo 4 

\secao{Métodos}

%% Nome do capitulo
\chapter{Metodologia}
% Label para referenciar
\label{cap:metodologia}
% Diminuir espacamento entre titulo e texto
\vspace{-1.9cm}

% Texto do capitulo
Primeiramente para esse projeto foram realizadas as pesquisas bibliogr�ficas. Trabalhos de outros autores que de alguma forma se relacionam com a monografia foram estudados com a finalidade fornecer uma base de conhecimento sobre o tema, como por exemplo,  o entendimento de todo o processo de descoberta do conhecimento e suas t�cnicas de minera��o de dados. Com isso foi poss�vel montar os cap�tulos de revis�o bibliogr�ficas abordando os principais conceitos utilizados durante o desenvolvimento.

No segundo momento, o desenvolvimento trata-se da aplica��o dos passos do processo de KDD sobre os Microdados do Censo da Educa��o Superior. Foram executados os seguintes passos: sele��o de dados, pr�-processamento, transforma��o, minera��o de dados e interpreta��o.Na sele��o de dados foi definida a fonte dos dados. No pr�-processamento removemos principalmente erros relacionados com a falta de informa��o. Na transforma��o os dados foram enriquecidos. Partindo para a minera��o de dados, os algoritmos de influencias, previs�o, clusteriza��o, e an�lise de cesta de compras foram aplicados. Finalmente, na interpreta��o dos dados os gr�ficos e an�lises encontradas s�o apresentados.
\capitulo{METODOLOGIA}

\secao{Modelo Tratado}

\secao{Proposta de Solução}

Será desenvolvido um sistema que otimiza a alocação das salas em até 90\% facilizando a vida do gerente.

\secao{O Sistema Desenvolvido}

\subsecao{Linguagens e Ferramentas Utilizadas}

\subsubsecao{Linguagens de Programação e Frameworks}

	%-----JAVASCRIPT
	JavaScript é uma linguagem de programação interpretada2 . Foi originalmente implementada como parte dos navegadores web para que scripts pudessem ser executados do lado do cliente e interagissem com o usuário sem a necessidade deste script passar pelo servidor, controlando o navegador, realizando comunicação assíncrona e alterando o conteúdo do documento exibido.
	É atualmente a principal linguagem para programação client-side em navegadores web. Foi concebida para ser uma linguagem script com orientação a objetos baseada em protótipos, tipagem fraca e dinâmica e funções de primeira classe. Possui suporte à programação funcional e apresenta recursos como fechamentos e funções de alta ordem comumente indisponíveis em linguagens populares como Java e C++.
	É baseada em ECMAScript padronizada pela Ecma international nas especificações ECMA-2623 e ISO/IEC 16262.

	%-----JAVA
	Java é uma linguagem de programação orientada a objeto desenvolvida na década de 90 por uma equipe de programadores chefiada por James Gosling, na empresa Sun Microsystems. Diferentemente das linguagens convencionais, que são compiladas para código nativo, a linguagem Java é compilada para um bytecode que é executado por uma máquina virtual. A linguagem de programação Java é a linguagem convencional da Plataforma Java, mas não sua única linguagem.


	%-----Play!
	The Play! Framework is a modern Java (and Scala) web application open-source framework that provides a clean alternative to bloated Enterprise Java stacks. Play has two version Play 1.x (Java & Scala) and Play 2.x (Java & Scala).
	Play is a high-productivity Java and Scala web application framework that integrates the components and APIs you need for modern web application development.
	Play is based on a lightweight, stateless, web-friendly architecture and features predictable and minimal resource consumption (CPU, memory, threads) for highly-scalable applications thanks to its reactive model, based on Iteratee IO.\par

	%-----Angular 
	AngularJS is an open-source JavaScript framework, maintained by Google, that assists with running single-page applications. Its goal is to augment browser-based applications with model–view–controller (MVC) capability, in an effort to make both development and testing easier.
	The library reads in HTML that contains additional custom tag attributes; it then obeys the directives in those custom attributes, and binds input or output parts of the page to a model represented by standard JavaScript variables. The values of those JavaScript variables can be manually set, or retrieved from static or dynamic JSON resources.

\subsubsecao{Sistema Gerenciador de Banco de Dados}

Postrgreesql

\subsubsecao{Ambiente de Desenvolvimento}

IDE eclipse, sublimeText, Google Chrome.

\subsecao{Modegem do Sistema}

\subsubsecao{Diagrama de Caso de Uso x}

\subsubsecao{Diagrama de Caso de Uso y}

\subsubsecao{Diagrama de Caso de Uso z}

\subsubsecao{Diagrama de Atividade x}

\subsubsecao{Diagrama de Classe x}

\subsubsecao{Diagrama de Classe y}

\subsubsecao{Diagrama de Classe z}

\subsubsecao{Modelagem de Dados}

\subsecao{Interface}

Twitter bootstrap.\par

\subsecao{Funcionalidades}

1.\par
2.\par
3.\par

\subsecao{Dados de Entrada}

Informados pelo gerente.

\subsecao{Funcionamento}

Processa os dados

\subsecao{Dados de Saída}

Grade de horarios completa

\subsecao{Considerações Finais do Capítulo}

Considerações finais do capitulo

\capitulo{RESULTADOS OBTIDOS}
	Despois do sistema implementado		

	Entrada processamento e saida

	ajuste do algoritomo de alocação

\capitulo{CONSIDERAÇÕES FINAIS}

\iniciocapitulo

*Relembra ro problema e comprar como o resultadoo bjetido.

Discussão dos resultados obtidos na pesquisa, onde se verificam as observações pessoais do autor. Poderá também apresentar sugestões de novas linhas de estudo. A conclusão deve estar de acordo com os objetivos do trabalho. A conclusão não deve apresentar citações ou interpretações de outros autores.

\secao{Trabalhos futuros}

Criar um DW para geração dos relatorios de acordo com a dimensão escolhida.

Utilização de outros algoritimos para a resolução do problema ex. algoritimos evolutivos formiga entre outros.

Pegar o feed back do usuario para melhoria na interface, e do algoritimo.

\referencias{Referencias}

\usepackage[titletoc,toc,page]{appendix}
\renewcommand{\appendixtocname}{Ap\'endices}
\renewcommand{\appendixpagename}{Ap\'endices}



\end{document}