\documentclass{article}
\usepackage[english]{babel}
\renewcommand{\familydefault}{\sfdefault}
\title{Title}
\author{Your Name}
 
\begin{document}
 
\maketitle{}
 
\section{Introduction}
 
This is where you will write your content.
 



tem um material bem interessante nessaspdflatextexify  monografias (gostei
principalmente do modo como foi feita a 3a, do Leonardo Ciscon). Note
que todas elas partem pela abordagem do algoritmo genético e tratam de
problemas grandes, então isso é algo com o que devemos nos preocupar:
qual o tamanho do problema do seu pai e se essa abordagem é válida, ou
se o sistema precisa de algo diferehttp://jabref.sourceforge.net/nte.

Ainda baseado no que eu vi delas, acho que podemos ratexify scunhar uma
estrutura assim:


1 - Introdução
Contexto: falar do timetable problem, falar de soluções conhecidas e
falar do seu problema específico de alocação de salas
Motivação/objetivos: explorar o porque da importância do projeto e
quais os objetivos (resultados) pretende-se atingir
Organização da monografia: é aquela parte formal onde descreve-se a
estrutura dos próximos capítulos

2 - Referencial teórico

Aqui precisamos definir qual será a abordagem adotada, mas de toda
forma creio que você já pode começar escrevendo sobre o problema da
alocação


2.1 - Alocação de horários

//Definição do problema, complexidade, exemplos de aplicação

Sistema para alocação de salas da ufmg


2.2 - Algoritmos/Heurísticas

Se a abordagem for heurística, deveríamos abordar alguns algoritmos
aqui, como o genético, "simulated annealing", tabu search, explicando
como podiam ser usados para resolver o problema.

3 - Metodologia

GitHub
MikTex
Sublime Text 2

Análise
RUP e UML

Dev
Playframework

Play is a high-productivity Java and Scala web application framework that integrates the components and APIs you need for modern web application development.

Play is based on a lightweight, stateless, web-friendly architecture and features predictable and minimal resource consumption (CPU, memory, threads) for highly-scalable applications thanks to its reactive model, based on Iteratee IO.

AngularJS

Why AngularJS?

HTML is great for declaring static documents, but it falters when we try to use it for declaring dynamic views in web-applications. AngularJS lets you extend HTML vocabulary for your application. The resulting environment is extraordinarily expressive, readable, and quick to develop.

Alternatives

Other frameworks deal with HTML’s shortcomings by either abstracting away HTML, CSS, and/or JavaScript or by providing an imperative way for manipulating the DOM. Neither of these address the root problem that HTML was not designed for dynamic views.

Extensibility

AngularJS is a toolset for building the framework most suited to your application development. It is fully extensible and works well with other libraries. Every feature can be modified or replaced to suit your unique development workflow and feature needs. Read on to find out how.

Sobre o PostgreSQL



Escrever o que foi decidido usar e porque. Documentação do sistema,
explicação das estruturas de dados e parâmetros do algoritmo


Os outros capítulos são referentes ao que for implementado; ou seja,
por enquanto creio que temos que nos preocupar com os acima. Veja que
não é pouco trabalho, hein...




Artigos

https://repositorio.ufsc.br/handle/123456789/78608
http://revistas.unicentro.br/index.php/ambiencia/article/viewArticle/123
http://revistaseletronicas.pucrs.br/ojs/index.php/hifen/article/view/4594
http://repositorio.ufu.br/handle/123456789/225
http://sbmac.org.br/tema/seer/index.php/tema/article/view/84


Links para pesquisa dos artigos

http://dl.acm.org/dl.cfm?CFID=246430573&CFTOKEN=96025967
http://ieee.org.br/news-item/vantagens-de-ser-um-membro-ieee/
http://scholar.google.com.br/

Para utilização da documentação do sistema
http://www.wthreex.com/rup/





\end{document}