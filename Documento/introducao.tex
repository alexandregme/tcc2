\sumario
\capitulo{INTRODUÇÃO}

\secao{Contextualização}

explicar o problema passo a passo

falar das restrições


O trabalho consiste na distribuição das disciplinas dos diversos cursos de graduação e pós-graduação apresentados pelos colegiados, em salas, atraves do sistema que será criado todo inicio de semestre assim que todos os colegiados já tenham enviado suas necessidades para aquele semestres.\par

Para realização da alocação, faz-se necessário o conhecimento das salas existentes para atender a demanda, alem de conhecer sua capacidade e suas características. A partir deste passo, faz-se necessário o levantamento das solicitações enviadas pelos colegiados, para alocar as disciplinas em salas adequadas de forma a atender todas as solicitações de forma otimizada.\par

O principal problema atualmente é a falta de salas adequadas para a distribuição das disciplinas, uma vez que as turmas atuais estão com um numero de alunos matriculados acima da capacidade das salas, devido a este problema o sistema deve propor um relatório de disciplinas que não atenderam a capacidade do prédio para que o responsável pela alocação possa negociar em prédios de outros cursos a alocação das disciplinas que não poderão ser alocadas no prédio que o sistema executou a alocação.\par


\secao{Objetivos}

O tratamento do problema de geração de horários em escolas carece de bons
trabalhos na literatura. Apesar de se encontrar ferramentas disponíveis, poucas tratam de
maneira eficiente as restrições reais existentes em escolas. Com este trabalho objetiva-se:

\subsecao{Objetivos Gerais}

O cap´ ıtulo a seguir descreve o sistema desenvolvido neste trabalho, que tem como objetivo
otimizar a alocac ¸˜ ao de disciplinas nas especificas salas do prédio de uma universidade, neste
caso o prédio da FAFICH UFMG, diretamente facilitando a vida do gerente. Por se tratar de um
problema especifico do local fica dificil encontrar tecnologias disponiveis para otimizac ¸˜ ao do
problema, neste o desenvolvimento de um sistema que atenda todas as necessidades exigidas é
de grande valia para o responsavel pela alocac ¸˜ ao.


	Este trabalho envolve conhecimentos de analise de sistemas e desenvolvimento almejando um sistema capaz tratar e otimizar a execução do problema de alocação de salas de uma universidade. Este trabalho possui um grande valor uma vez que pode facilita a vida do responsável pela alocação das salas, por se tratar de um trabalho manual, trabalhoso e que para a execução são necessárias em torno de trinta horas que poderiam estar sendo utilizadas para uma tarefa mais importante.\par

\subsecao{Objetivos Específicos}

	Objetivos específicos:\par

	- Desenvolvimento do sistema.\par

	- Apresentação de um modelo matemático acerca do problema;

	- Implementação de um algoritmo que solucione o modelo apresentado considerandoas restrições mais comuns para a geração de um horário de qualidade.

	- Otimizar o tempo do gestor.\par

	- Eficiência na geração dos relatórios.\par

\secao{Justificativa}

\secao{Organização do Trabalho}

Este trabalho está definido da seguinte forma, foi dividido em seis capítulos, sendo este capítulo 1 e mais cinco outros.\par

O capítulo 2 apresenta o referencial teórico do trabalho, descrevendo os conceitos utilizados para o desenvolvimento do projeto proposto: conceitos de TimeTable e Heurísticas.\par

No capítulo 3 é apresentada a metodologia do sistema e as tecnologias adotadas para desenvolvimento da solução.\par

No capítulo 4 iremos descrever e citar detalhadamente as características e propostas de desenvolvimento do sistema desenvolvido, proposto para este trabalho.\par

A conclusão deste trabalho e considerações finais são mostrados nos capítulos 5 e 6.\par

Se tiver anexo explicar cada anexo.
