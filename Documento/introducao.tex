\sumario
\capitulo{INTRODUÇÃO}

\iniciocapitulo

O trabalho consiste na distribuição das disciplinas dos diversos cursos de graduação e pós-graduação apresentados pelos colegiados, em salas, atraves do sistema que será criado todo inicio de semestre assim que todos os colegiados já tenham enviado suas necessidades para aquele semestres.\par

Para realização da alocação, faz-se necessário o conhecimento das salas existentes para atender a demanda, alem de conhecer sua capacidade e suas características. A partir deste passo, faz-se necessário o levantamento das solicitações enviadas pelos colegiados, para alocar as disciplinas em salas adequadas de forma a atender todas as solicitações de forma otimizada.\par

O principal problema atualmente é a falta de salas adequadas para a distribuição das disciplinas, uma vez que as turmas atuais estão com um numero de alunos matriculados acima da capacidade das salas, devido a este problema o sistema deve propor um relatório de disciplinas que não atenderam a capacidade do prédio para que o responsável pela alocação possa negociar em prédios de outros cursos a alocação das disciplinas que não poderão ser alocadas no prédio que o sistema executou a alocação.\par

\secao{Contextualização}

explicar o problema passo a passo

falar das restrições


\secao{Objetivos}

O tratamento do problema de geração de horários em escolas carece de bons
trabalhos na literatura. Apesar de se encontrar ferramentas disponíveis, poucas tratam de
maneira eficiente as restrições reais existentes em escolas. Com este trabalho objetiva-se:


\subsecao{Objetivos Gerais}


	Este trabalho envolve conhecimentos de analise de sistemas e desenvolvimento almejando um sistema capaz tratar e otimizar a execução do problema de alocação de salas de uma universidade. Este trabalho possui um grande valor uma vez que pode facilita a vida do responsável pela alocação das salas, por se tratar de um trabalho manual, trabalhoso e que para a execução são necessárias em torno de trinta horas que poderiam estar sendo utilizadas para uma tarefa mais importante.\par

Com este trabalho objetiva-se:\par

	- Desenvolvimento do sistema.\par
	- Apresentação de um modelo de algorítimo que atenda o problema proposto.\par
	- Otimizar o tempo do gestor.\par
	- Eficiência na geração dos relatórios.\par

	? Apresentação de um modelo matemático acerca do problema;
? Implementação de um algoritmo que solucione o modelo apresentado considerando
as restrições mais comuns para a geração de um horário de qualidade;
? Consideração, de forma simultânea, de mais duas importantes restrições: a
eliminação de janelas e de aulas isoladas no horário do professor;
? Desenvolvimento de uma interface gráfica amigável que utilize este algoritmo

\subsecao{Objetivos Específicos}

\secao{Justificativa}

\secao{Organização do Trabalho}

Este trabalho está definido da seguinte forma, foi dividido em cinco capítulos, sendo este capítulo 1 e mais quatro outros.\par

O capítulo 2 apresenta o referencial teórico do trabalho, descrevendo os conceitos utilizados para o desenvolvimento do projeto proposto: conceitos de ---------------------------------------\par

No capítulo 3 é apresentada a metodologia do sistema e as tecnologias adotadas para desenvolvimento da solução.\par

No capítulo 4 iremos descrever e citar detalhadamente as características e propostas de desenvolvimento do sistema -------, proposto para este trabalho.\par

A conclusão deste trabalho e planos futuros são mostrados no Capítulo 5.\par

Se tiver anexo explicar cada anexo.
