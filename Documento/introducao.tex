\capitulo{INTRODUÇÃO}

\iniciocapitulo
A introdução deve conter a natureza do trabalho, justificativa, objetivos, o tema proposto e outros elementos para situar o trabalho.\par

\secao{Considerações Iniciais}
Escrever sobre os objetivos e as justificativas

\secao{Objetivos e Justificativas}
Escrever sobre os objetivos e as justificativas

\secao{Organização do Trabalho}
Este trabalho está definido da sequinte forma. Capitulo 1 Capitulo 2 Capitulo 3 Capitulo 4 

% Exemplo de tabela:

\begin{table}[!htb]
	\caption[Descrição na Lista de Tabelas]{Legenda da tabela}
	\centering
	\label{tab:tabela}
	\begin{tabular}{c|c|c}
		\hline
		Coluna 1 & Coluna 2 & Coluna 3 \\
		\hline
 		Célula 1 & Célula 2 & Célula 3 \\
		Célula 4 & Célula 5 & Célula 6 \\
		Célula 7 & Célula 8 & Célula 9 \\
		\hline
	\end{tabular}
	\\ \textbf{\footnotesize Fonte: Fonte da tabela}
\end{table}

% Exemplo de figura:

\begin{figure}[!htb]
   \caption[Descrição na Lista de Figuras]{Legenda da figura}
   \label{fig:figura1}
   \centering
   \includegraphics{LogoPUC.jpg}
   \\ \textbf{\footnotesize Fonte: Fonte da figura}
\end{figure}
