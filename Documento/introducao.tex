\sumario
\capitulo{INTRODUÇÃO}

\secao{Contextualização}

Instituições de ensino universitarias se deparam todo início de semestre letivo com o problema de alocação de salas, este problema pode ser definido como \textit{Classroom Assignment} que é uma instancia do \textit{course timetabling}, problemas desta instancia são prolemas de otimização combinatoria. Problemas de otimização combinatoria tem a complexidade NP-difícil, para a resolução de problemas desta complexidade em um tempo razoavel são propostas algumas tecnicas denominadas meta-heuristicas. Esta técnicas amenizam a dificildade da resolução destes problemas para encontrar uma solução em um tempo habil, uma vez que, a resolução destes problemas de forma manual é de grande dificuldade em alguns casos pode demandar semanas de trabalho da pessoa responsável.\par


Em suma o trabalho consiste na distruições das disciplinas pertecentes aos cursos de graduação e pós-graduação apresentados po alguns  colegiados da Universidade Federal de Minas Gerais (UFMG) , em salas disponibilizadas pelo predio da Faculdade de Filosofia e Ciências Humandas (FAFICH). Para a distribuição destas disciplinas nas salas, foi criado um sistema que utiliza conceitos de algorítimo genético para resolver o problema PAS. Esta demanda de alocação acontece todo início de semestre e é executada assim que todos os colegiados tenham enviados suas solicitações necessidas para aquele semestre.\par

Uma vez que os termos da biologia utilizados foram lincados com o problema, o sistema gera uma alocação com grandes chances de atender as necessidades da instituição. \par

\secao{Objetivos}

O tratamento do problema de alocação de salas em instituições de ensino carece de bons trabalhos na literatura. Apesar de se encontrar ferramentas disponíveis, poucas tratam de maneira eficiente as restrições reais existentes nas instiruições. Com este trabalho objetiva-se:

\subsecao{Objetivos Gerais}

O objetivo deste trabalho é utilizar os conceitos de algoritimo genético para a resolução dos problemas denominados PAS através do desenvolvimento de um sistema que atenda todas as necesssidades da instiruição e facilite o gerenciamento das informações da instituição como salas, disciplinas e demais informações.

\subsecao{Objetivos Específicos}

	- Desenvolvimento do sistema.\par

	- Implementação de um algoritmo que proporcione uma solução de qualidade.

	- Otimizar o tempo do gestor.\par

	- Eficiência na geração dos relatórios.\par

\secao{Justificativa}

A solução de problemas de PAS através de meta-heurísticas se trata de uma área ainda não consolidada, por mais que existam trabalhos relacionados ao tema espera-se que as conclusões realizadas neste trabalho agreguem valor algum falor para os trabalhos futuros.

\secao{Organização do Trabalho}

Este trabalho está definido da seguinte forma, foi dividido em seis capítulos, sendo este capítulo 1 e mais cinco outros.\par

O capítulo 2 apresenta o referencial teórico do trabalho, descrevendo os conceitos utilizados para o desenvolvimento do projeto proposto: conceitos de TimeTable e Heurísticas.\par

No capítulo 3 é apresentada a metodologia do sistema e as tecnologias adotadas para desenvolvimento da solução.\par

No capítulo 4 iremos descrever e citar detalhadamente as características e propostas de desenvolvimento do sistema desenvolvido, proposto para este trabalho.\par

A conclusão deste trabalho e considerações finais são mostrados nos capítulos 5 e 6.\par