\capitulo{REFERENCIAL TEÓRICO}

\secao{Problemas de Otimização}

Otimização é o processo de encontrar a melhor solução, também chamada de solução ótima para um determinado problema \cite{timoteo2005desenvolvimento}.\par

De acordo com \cite{steiglitz1982combinatorial} a constituição de um problema de otimização se deve aos termos vizinhança, ótimo local e ótimo global. O termo vizinhança se trata de um subconjunto do conjunto de soluções do problema, ótimo local pode ser tratado como o melhor resultado em uma vizinhança, e ótimo global é a melhor solução encontrada no conjunto de acordo com a função objetivo.\par

De acordo com a figura 1 pode se observar a relação entre ótimo local e ótimo global em conjunto de soluções de um problema típico de minimização.\par

\begin{figure}[!htb]
\caption[Representação de um problema de minimização com ótimos locais]{Representação de um problema de minimização com ótimos locais}
\label{fig:figura2}
\centering
\includegraphics[scale=0.55]{imagens/problemaOtimizacao.png}
\\ \textbf{\footnotesize Fonte: \cite{timoteo2005desenvolvimento}}
\end{figure}
	
Segundo \cite{steiglitz1982combinatorial}, os problemas de otimização são divididos em duas categorias, problemas com variáveis contínuas e problemas com variáveis discretas. Problemas com variáveis discretas também podem ser conhecidos como Problemas de Otimização Combinatória (POC).\par

Conforme cita \cite{raupp2003introduccao}, o problema de otimização combinatória pode ser denominado como a ação de maximizar ou minimizar uma função objetiva de diversas variáveis sujeita a um conjunto de restrições, dentro de um contexto.\par

De acordo com \cite{opac-b1092847} problemas do tipo POC tratam do estudo matemático para encontrar agrupamentos, arranjos ou a seleção ótima de objetos discretos, logo, não permitindo, a utilização de métodos clássicos de otimização contínua para sua resolução.\par


Segundo \cite{golbarg2000otimizaccao} a ocorrência de problemas de otimização combinatória podem acontecer em diversas áreas, projetos de sistemas de distribuição de energia elétrica, posicionamento de satélites, roteamento ou escalonamento de veículos, sequenciamento de genes e DNA, classificação de plantas e animais.\par

De acordo com \cite{deleonardo} em problemas de otimização combinatória, cujo universo de dados é grande e existe um grande número de combinações, o que torna inviável a análise de todas soluções possíveis em um tempo adequado, utilizamos as heurísticas, também conhecidas como algoritmos heurísticos, que são métodos que compõem uma gama de soluções para problemas de otimização combinatória.

\secao{Heurística}

O termo heurística é derivado do grego \textit{heuriskein}, o que significa descobrir ou achar. De acordo com \cite{timoteo2005desenvolvimento} o significado da palavra em pesquisa operacional vai um pouco além da raiz etimológica. Segundo \cite{steiglitz1982combinatorial}, heurísticas são consideradas métodos de aproximação ou métodos de busca de solução, deve se levar em consideração que não exista uma garantia formal de seu desempenho e uma garantia de que estas heurísticas que iram encontrar uma solução. As heurísticas, apesar de não garantirem encontrar a solução ótima para um problema, procuram por soluções consideradas de boa qualidade em um tempo computacional razoável.\par

Segundo \cite{evans1992optimization} heurísticas são necessárias para implementação de problemas NP Difícil, caso deseje-se resolver tais problemas em um tempo  razoável.\par

Ressalta-se que dentre as heurísticas, as chamadas meta-heurísticas, merecem especial atenção pois adotam técnicas para amenizar, a dificuldade que os métodos heurísticos têm de escapar dos ótimos locais. As meta-heurísticas podem partir em busca de regiões mais promissoras no espaço de soluções, alem disto, as meta-heurísticas possuem grande abrangência, podendo ser aplicada à maioria dos problemas de otimização combinatória.\cite{nascimento2005aplicaccao}\par

As meta-heurísticas surgiram como uma alternativa para amenizar a dificuldade que os métodos heurísticos tem de escapar dos chamados ótimos locais.\cite{nascimento2005aplicaccao}

Segundo \cite{adrianocesar} uma heurística é a instanciação de uma meta-heurística, ou seja, a aplicação da mesma em um problema específico de otimização.\par

Como exemplos de meta-heurísticas temos Busca Tabu (\textit{Tabu Search}), Otimização por Colônias de Formigas (\textit{Ant Colony Optimization}), Recozimento Simulado (\textit{Simulated Annealing}) e Algoritmo Genético (\textit{Genetic Algorithm}).\par

\subsecao{Busca Tabu}

Busca tabu (BT) é uma meta-heurística adaptativa, que utiliza uma estrutura de memória através de uma lista, contendo um histórico de evolução para evitar que o processo de busca forme ciclos, ou seja, o retorno a um ótimo local previamente visitado \cite{souza2000} , \cite{armentanointroduccao} e \cite{subramanian2006aplicaccao}.\par


Segundo \cite{subramanian2006aplicaccao} BT foi desenvolvida por \cite{glover1986future} com o objetivo de encontrar soluções para problemas de programação linear. Ao formalizar a técnica, o autor publicou uma serie de trabalhos envolvendo diversas aplicações da meta-heurística. \par

Basicamente o funcionamento do BT a feito partir da definição de uma população inicial ${S_0}$, o algoritmo explora cada iteração, de um subconjunto V da vizinhança N(S) da solução corrente S. O membro S’ de V com melhor valor nessa região segundo a função f(.) torna-se a nova solução corrente mesmo que S’ seja pior que S isto é, que f(S’) $>$ f(S) para um problema de minimização\cite{souza2000}. A figura 2 representa o pseudocódigo do algorítimo da Busca Tabu.

Segundo \cite{armentanointroduccao}  o algorítimo tem um intensivo uso de memória o que é uma característica essencial deste método. Para o autor o uso da memória pode ajudar a intensificar a busca em regiões com grande chances de se encontrar o resultado, ou até mesmo, diversificar a busca através de regiões inexploradas.
\begin{figure}[!htb]
\caption[Representação do pseudocódigo do algorítimo da Busca Tabu]{Representação do pseudocódigo do algorítimo da Busca Tabu}
\label{fig:figura2}
\centering
\includegraphics[scale=0.50]{imagens/representacaoBuscaTabu.png}
\\ \textbf{\footnotesize Fonte: \cite{souza2000}}
\end{figure}

De acordo com \cite{armentanointroduccao}  devemos adotar alguns procedimentos para que o processo de busca tenha um melhor resultado, listas tabu dinâmicas, passagens por regiões planas, intensificação, diversificação,\textit{path relinking}. Listas tabu dinâmicas tem como objetivo evitar que o algorítimo entre em processo de ciclo.Passagens por regiões planas pode levar o algorítimo a pensar que não existem melhoras significativas na qualidade das soluções e atingir o critério de parada, para evitar esta situação é necessário aumentar o tamanho da lista tabu enquanto o algorítimo estiver passando pela região plana e voltar a reduzir quando houver mudança no valor da função objetiva.Intensificação são técnicas utilizadas para concentrar os esforços da pesquisa em regiões consideradas promissoras. Diversificação é uma técnica que utiliza memória de longo prazo para redirecionar a pesquisa para regiões que ainda não foram suficientemente exploradas. PathRelinking trada ta intensificação de incorporar atributos de soluções de boa qualidade (chamadas de soluções elite), em seguida explora caminhos que contenham uma ou mais soluções de elite.

Ainda segundo \cite{armentanointroduccao} uma característica importante do método é que a solução final tem pouca ou nenhuma dependência da escolha feita para a solução inicial, isso graças aos mecanismos implementados pelo método, que fogem de ótimos locais.

\subsecao{Recozimento Simulado}

Técnica de busca local probabilística, proposta originalmente por \cite{kirkpatrick1983optimization}, que se fundamenta em uma analogia com a termodinâmica, ao simular o 
resfriamento de um conjunto de átomos aquecidos.\par 

Isto é, conforme \cite{noronha2003abordagem} em analogia a física da matéria: levando um cristal a sua temperatura de fusão, as moléculas estão desordenadas e se agitam livremente. Ao resfriar-se a amostra de maneira infinitamente lenta, as moléculas vão adquirir a estrutura cristalina estável que tem um nível de energia mais baixo possível.\par 

Segundo \cite{souza2002experiencias} o processo se inicia com um membro qualquer do espaço de soluções, normalmente gerado aleatoriamente, e seleciona um de seus vizinhos randomicamente. Se este vizinho for melhor que o original ele é aceito e substitui a solução corrente. Se ele for pior por uma quantidade ∆, ele é aceito com uma probabilidade e -∆/T , onde T decresce gradualmente conforme o progresso do algoritmo. Esse processo é repetido até que T seja tão pequeno que mais nenhum movimento seja aceito. A melhor solução encontrada durante a busca é tomada como uma boa aproximação para a solução ótima. Originalmente, Simulated Annealing foi derivado de simulações em termodinâmica e por esta razão o parâmetro T é referenciado como temperatura e a maneira pela qual ela é reduzida é chamada de processo de resfriamento.


A figura 3 representa o pseudocódigo do algoritmo Simulated Annealing.

\begin{figure}[!htb]
\caption[Representação do pseudocódigo do algorítimo Simulated Annealing]{Representação do pseudocódigo do algorítimo Simulated Annealing}
\label{fig:figura2}
\centering
\includegraphics[scale=0.55]{imagens/representacaoSimulatedAnnealing.png}
\\ \textbf{\footnotesize Fonte: \cite{souza2002experiencias}}
\end{figure}

Conforme \cite{aarts1988simulated} a analogia com a otimização (combinatória ou não) é bastante direta. Os estados da matéria são as soluções realizáveis, a quantidade objetiva substitui a energia, os estados metaestáveis da matéria sendo ótimos locais e a estrutura cristalina corresponde ao ótimo global.\par 


\subsecao{Algoritmos genéticos}

%o que é 

De acordo com \cite{goldberg1989genetic} algoritmos genéticos são baseados na teoria da evolução das espécies elaborada por \cite{darwin1968origin} utilizando os conceitos da biológia tais como genes, individuo, população, cromossomos, cruzamento, mutação e seleção. Estes algoritmos foram introduzidos por \cite{holland1975adaptation} para resolver problemas chamados \textit{timetabling}.

Para entender melhor estes termos \cite{mitchell1998introduction} cita os principais termos biológicos importantes para o entendimento do funcionamento dos algoritmos genéticos.\par
Gene se trada de uma característica particular de um cromossomo. Um cromossomo é composto por um ou mais genes pode se dizer também que é uma sequencia de genes que será caracterizada como a solução do problema. \textit{Fitness} significa a aptidão do indivíduo em um determinado ambiente. Individuo é a combinação do cromossomo mais o \textit{fitness} calculado através da função objetiva.População é um grupo de indivíduos.\par

Em seu trabalho \cite{lucas2000algoritmos} descreve algoritmo genéticos da seguinte forma. São algoritmos que trabalham sobre uma população, que através de uma função de adaptação aconteça a evolução, primeiramente é inicializada uma população, logo após acontece iram acontecer os processos de seleção reprodução e mutação, os mesmos ocorreram a cada geração até que os critérios de parada aconteçam. Afirma que os termos utilizados pertencem à tradição existente no meio da computação evolutiva de utilizar, com certa liberdade os termos da biologia.

A Figura~\ref{fig:ag} apresentada no trabalho de \cite{lucas2000algoritmos} mostra a estrutura funcional de um algoritmo genético tradicional.\par
Primeiramente deve ser feita a inicialização básica da população, são utilizadas funções aleatórias para gerar os indivíduos assim teremos uma biodiversidade na população, podemos utilizar varias alternativas para esta tarefa: Inicialização randômica uniforme onde cada gene recebe um valor sorteado de forma aleatória e uniforme. Inicialização randômica não uniforme alguns valores para se preencher o gene são escolhidos com mais frequência que os demais. Inicialização randômica com "dope" alguns indivíduos criados são otimizados e adicionados a população aleatória.

Para cada indivíduo de uma população é realizada um processo de avaliação o que representa o seu grau de adaptação no ambiente. Este é o primeiro passo da seleção que pode ocorrer de varias formas.\par

Seleção roda de roleta é um método tradicional, para cada indivíduo é atribuído um espaço na roleta sendo o tamanho proporcional ao valor da aptidão do individuo. Esta role gira N vezes onde N é o numero é o tamanho da população selecionando assim os pais para próxima geração. Seleção por torneio são selecionados indivíduos da população anterior e escolhidos os dois que que contêm o maior valor de aptidão.\par

Cruzamento também conhecido como \textit{crossover} é um operador genético onde é selecionado um ponto de corte após isto é realizada a troca genética entre os subconjutos de genes dos pais criando um filho contendo material genético dos mesmos.

Mutação consiste em mudar aleatoriamente um indivíduo para criar um novo indivíduo com um diferente material genético, este mecanismo serve para que exista uma variabilidade genética e para que o algorítimo não fique preso em um máximo ou minimo local.

Geração corresponde a uma iteração realizada sobre a população.

Elitismo pode ser definido como o operador genético que seleciona os melhores indivíduos de uma população com isto garantindo com que não haja perda da melhor solução encontrada.

\begin{figure}[!htb]
\caption[Estrutura funcional de um algoritmo genético tradicional]{Estrutura funcional de um algoritmo genético tradicional}
\label{fig:ag}
\centering
\includegraphics[scale=0.62]{imagens/ag.png}
\\ \textbf{\footnotesize Fonte: \cite{lucas2000algoritmos}}
\end{figure}

 Estrutura de um Algoritmo Genético 
 
Na maioria dos casos um algoritmo genético é estruturado da seguinte forma 
(adaptado de Sivanandam e Deepa, 2008): 
Início – Gera-se uma população aleatória de n cromossomos. Quando já se tem 
alguma informação sobre o problema, pode-se introduzir soluções adequadas para já iniciar a 
população com um valor de aptidão alto ou adequado. 
Aptidão – é associado a cada cromossomo o seu valor de aptidão na população. 
Seleção – os pais são selecionados de acordo com o seu valor de aptidão, quanto 
melhor o valor de aptidão, maior a chance desses pais serem selecionados. 
Cruzamento (crossover) – de acordo com a probabilidade de ocorrer o cruzamento, 
as informações entre os pais são trocadas, formando-se os filhos. Se a operação de 
cruzamento não acontecer, os filhos serão exatamente iguais aos pais (dependendo ainda se a 
mutação ocorrerá ou não). 
Mutação – com certa probabilidade, geralmente bem menor do que a probabilidade de 
ocorrer o cruzamento, a mutação pode ocorrer em cada locus dos descendentes. 
Aceitação – os filhos são colocados na nova população. 
Troca – substitui-se a antiga população pela nova população. 
Teste – Se a condição de parada é satisfeita, o programa deve retornar a melhor 
solução encontrada na população atual. 
Laço – Se não, volte para o passo em que a aptidão é calculada. 


Segundo \cite{oliveira2005algoritmo}, o processo de evolução executado por um algoritmo genético corresponde a um procedimento de busca no espaço de soluções potenciais para o problema e, como enfatiza \cite{michalewicz1996evolutionary}, esta busca requer um equilíbrio entre dois objetivos aparentemente conflitantes: a procura das melhores soluções na região que se apresenta promissora ou fase de intensificação e a procura de outra região ou exploração do espaço de busca, também conhecida como diversificação.\par


Segundo \cite{hamawaki2011geraccao} e \cite{oliveira2005algoritmo} algoritmos genéticos são eficientes para encontrar soluções ótimas ou quase ótimas, pois as limitações são minimas dos demais métodos de busca tradicionais.


Conforme cita \cite{oliveira2005algoritmo}, os algoritmos genéticos foram introduzidos por \cite{holland1975adaptation}, com intuito de aplicar a teoria da evolução das espécies elaborada por \cite{darwin1968origin} utilizando os conceitos da evolução biológica como genes, cromossomos, cruzamento, mutação e seleção na computação procurando explicar rigorosamente processos de adaptação em sistemas naturais e desenvolver sistemas artificiais (simulados em computador) que mantenham os mecanismos originais, encontrados em sistemas naturais.\par





%historia


%algoritimo

Cofificação do indiviruo 

sorteio e aleatoriedades

base de dados

criação da população inicial

calculo da aptidao

operadores geneticos

seleção 
elitismo 
cruzamento
mutação
evolução


%outros

A computação 





Segundo \cite{oliveira2005algoritmo}, o processo de evolução executado por um algoritmo genético corresponde a um procedimento de busca no espaço de soluções potenciais para o problema e, como enfatiza \cite{michalewicz1996evolutionary}, esta busca requer um equilíbrio entre dois objetivos aparentemente conflitantes: a procura das melhores soluções na região que se apresenta promissora ou fase de intensificação e a procura de outra região ou exploração do espaço de busca, também conhecida como diversificação.\par

Ainda segundo \cite{oliveira2005algoritmo}, os algoritmos genéticos têm se mostrado ferramentas poderosas para resolver problemas onde o espaço de busca é muito grande e os métodos convencionais se mostraram ineficientes.\par


•  Avaliação das soluções: os algoritmos genéticos utilizam informações de custo ou recompensa penalizando ou premiando determinadas características das soluções; \par

•  Regras: os algoritmos genéticos utilizam regras probabilísticas e não determinísticas; \par

O algoritmo genético é uma forma da estratégia gerar-e-testar realizando os testes baseados nos parâmetros da evolução biológica. Uma desvantagem notável é a variação dos operadores genéticos do algoritmo em cada problema. Dessa forma, para resolução de determinado problema, torna-se necessário um estudo particular a respeito do mesmo. \par

O algoritmo genético atua sobre uma população fazendo com que esta evolua de acordo com uma função de avaliação. O funcionamento é iterativo iniciando com a geração de uma população inicial que pode ser aleatória ou não, seguida do processo de avaliação, seleção, cruzamento e mutação, que ocorre a cada iteração até que seja atingido algum critério de parada. Os passos gerais de um algoritmo genético são ilustrados na figura 
Figura XXXX. Cada passo pode ser realizado de várias maneiras e pode variar de problema para problema \cite{timoteo2005desenvolvimento}.\par 

Figura Etapas de um Algoritmo Genético Básico 

\secao{TimeTabling}

Segundo \cite{kripkasimulated} os problemas de programação de horários (PPH), também conhecidos como \textit{TimeTabling}, são os problemas que mais se destacam nas organizações acadêmicas. De acrodo com \cite{schaerf1999survey} estes problemas são divididos em três categorias \textit{school timetabling}, \textit{course timetabling} e \textit{examination timetabling}.\par

\textit{School TimeTabling}: Se trata basicamente da geração de horários semanais, em escolas de segundo grau, onde deve-se evitar os choques entre os horários das disciplinas e que cada professor receba apenas uma turma para cada horário. Neste caso o aluno recebe um número fixo de disciplinas a serem cursadas.\par

\textit{Course TimeTabling}: Diz respeito à alocação de aulas de uma universidade típica. Neste problema os alunos podem escolher as matérias em que vão se matricular, portando o problema tem como objetivo minimizar os possíveis choques entre as disciplinas, professores e horários disponibilizados pela instituição de ensino.\par

\textit{Examination TimeTabling}: Aborda o problema de programação de horários dos exames da instituição, de maneira que, disciplinas que tenham alunos em comum, distanciem o máximo possível as datas dos exames.\par

Segundo \cite{pinheiro2001ambiente} o problema de programação de horários vem sendo abordado desde a década de 60, sendo que os primeiros trabalhos a se destacarem foram realizados na década de 80.\par

O Problema de Alocação de Salas (PAS) também conhecido como \textit{Classroom Assignment} é tratado como parte do problema de programação de cursos universitários \textit{course timetabling}. Segundo \cite{marinho2005heuristicas} varias instituições universitárias se deparam com o PAS durante o início de cada semestre letivo, este problema é considerado NP-Difícil por \cite{even1975complexity} e \cite{carter1992classroom}, com isto, a determinação da solução ótima do problema, em um período de tempo aceitável se torna uma tarefa difícil.\par
Segundo \cite{kripkasimulated} o problema deve considerar que as disciplinas dos cursos universitários já tenham seus horários de início e de término definidos. O problema se resume então na alocação das disciplinas às salas desta universidade respeitando os horários destas disciplinas e as demais restrições exigidas.

Segundo \cite{souza2000} boa parte das universidades ainda resolvem este problema de forma manual, o que torna o processo árduo e demorado, podendo levar vários dias para ser concluído.\par

Uma vez que é de extrema dificuldade encontrar a solução ótima do PAS em tempo razoável, este problema é normalmente tratado através de técnicas heurísticas, que apesar de não garantirem encontrar a solução ótima do problema, são capazes de retornar uma solução de qualidade em um tempo adequado.\cite{nascimento2005aplicaccao}.Segundo \cite{even1975complexity} o PAS pertence a classe de Problemas de Otimização Combinatória (POC).\par


\secao{Trabalhos Relacionados}
trabalho do marinho que usa tabu.
trabalho da silvia que usa Recozimento Simulado (Simulated Annealing).
trabalho da leonardo que usa Algoritmos Genéticos (AG).
achar algum trabalho que utiliza o algoritimo das formigas.
falar porque o trabalho do cara se assemelha ao meu trabalho.