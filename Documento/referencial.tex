\capitulo{REFERENCIAL TEÓRICO}

\secao{TimeTable}

%Explicar o que é

%tipos
\cite{souca(2000)}Muitas variantes do problema têm sido propostas na literatura, e diferem umas das outras pelo tipo de instituição de ensino envolvida, universidades ou escolas médias, e pelo tipo de restrições impostas ao problema. \cite{Schaerf (1999)} cita três classes de problemas:


\textit{Examination Timetabling}: seqüenciamento de exames de um conjunto de cursos em uma universidade, evitando exames simultâneos de cursos com estudantes em comum, e espalhando os exames o máximo possível. Segundo \cite{Souza (2000)}, apesar da similaridade com o course timetabling, eles se distinguem, sobretudo pela natureza das restrições envolvidas. Entre as restrições típicas deste tipo de problemas, destaca-se: nenhum estudante pode fazer mais do que certo número de exames por dia, exames de certas disciplinas não podem preceder a exames de determinadas disciplinas, alguns exames têm que ser realizados em um mesmo horário.\par

\textit{School Timetabling}: seqüenciamento semanal das aulas de uma escola, evitando que professores e alunos tenham mais de uma aula simultaneamente. Basicamente, existe um conjunto de turmas, um conjunto de professores e um conjunto de horários reservados para a realização das aulas.\par

\textit{Course Timetabling}: diz respeito à alocação de aulas de uma universidade típica. Basicamente há um conjunto de disciplinas (Cálculo I, Engenharia de Software, Genética, etc.) e para cada disciplina um número de aulas. Há, também, um conjunto de cursos (Engenharia de Alimentos, Ciência da Computação, Agronomia, etc). Cada curso envolve um conjunto de disciplinas. Os alunos matriculam-se em turmas das disciplinas de seu curso. Uma turma de uma disciplina pode ter estudantes de cursos diferentes. Há, por último, um conjunto de horários disponibilizados para a realização das aulas. O problema, então, trata do seqüenciamento semanal das aulas evitando a simultaneidade de disciplinas e respeitando os horários disponibilizados.\par

%Complexidade

O problema de alocação de salas denotado por PAS é um problema NP-Difícil(\cite{Even et al. 1976} e \cite{Carter 1992}).\par

Segundo \cite{marinho} diversas instituições universitárias se deparam com o PAS durante o início de cada semestre letivo. Boa parte destas ainda resolve tal problema manualmente, o que torna o processo árduo e demorado, podendo levar vários dias para ser concluído.Ainda segundo \cite{souza} a elaboração de um quadro de horários por esta via pode demandar duas semanas de trabalho em uma escola secundária ou até um mês em uma universidade e esta solução obtida pode ser insatisfatória com respeito a diversos aspectos.\par

%algo a mais sobre o PAS

\cite{Schaerf (1999)} e \cite{Werra (1985)} acreditam que problemas de geração de horários não podem ser completamente automatizados. Há duas justificativas para isso: por um lado, há razões que não podem ser facilmente expressas em um sistema automatizado, que tornam um quadro de horário melhor que o outro. Por outro, uma vez que o espaço de soluções é vasto, a intervenção humana pode conduzir a busca em direção a regiões promissoras, nas quais o sistema, por si só, dificilmente teria condições de chegar.\par

Uma vez que não é possível encontrar a solução ótima do PAS em tempo razoável, esse problema é normalmente tratado através de técnicas heurísticas e algoritmos aproximativos que apesar de não garantirem encontrar a solução ótima do problema são capazes de retornar uma solução de qualidade em um tempo adequado para as necessidades da aplicação.\par

%onde pode ser usado

\secao{Heurística}


Segundo \cite{Papadimitriou & Steiglitz (1982)}, as heurísticas são quaisquer métodos de 
aproximação sem uma garantia formal de seu desempenho. As heurísticas, apesar de não garantirem encontrar a solução ótima para um problema, procuram por soluções consideradas de boa qualidade em um tempo computacional razoável.\par

De acordo com \cite{P APADIMITRIOU & S TEIGLITZ (1982)}, as heurísticas são quaisquer métodos de aproximação sem uma garantia formal de seu desempenho. Sendo necessárias para implementação de problemas NP Difícil − , caso deseje-se resolver tais problemas em um tempo computacional razoável \cite{(E VANS & M INIEKA , 1978)}.\par

O termo heurística é derivado do grego heuriskein, que significa descobrir ou achar. Mas o significado da 
palavra em pesquisa operacional vai um pouco além da raiz etimológica. De um modo geral, o sentido dado ao termo heurística, refere-se a um método de busca de soluções em que não existe qualquer garantia de sucesso.\par


\cite{leonardo}Em Problemas de Otimização Combinatória, cujo universo de dados é grande, existe um número muito extenso de combinações, tornando inviável a análise de todas possíveis soluções, visto que o tempo computacional para uma enumeração completa seria demasiadamente longo. Neste sentido, têm-se as heurísticas, também conhecidas como algoritmos heurísticos, que são métodos que compõem uma gama relativamente nova de soluções para Problemas de Otimização Combinatória. 


Ressalta-se que dentre as heurísticas merecem especial atenção as chamadas meta-heurísticas que adotam técnicas para amenizar a dificuldade que os métodos heurísticos têm de escapar dos chamados ótimos locais. As meta-heurísticas podem partir em busca de regiões mais promissoras no espaço de soluções. As meta-heurísticas possuem grande abrangência, podendo ser aplicada à maioria dos problemas de otimização combinatória.\par

Podem-se citar como exemplo as meta-heurísticas: Busca Tabu (\textit{Tabu Search}), Otimização por Colônias de Formigas (\textit{Ant Colony Optimization}), Recozimento Simulado (\textit{Simulated Annealing}) e Algoritmo Genético (\textit{Genetic Algorithm}). Uma heurística é a instanciação de uma meta-heurística, ou seja, a aplicação da mesma em um problema específico de otimização.\par


\subsecao{Busca Tabu}

A metaheurística BT foi inicialmente desenvolvida por \cite{Glover (1986)} como uma proposta de
solução para problemas de programação inteira. A partir de então, o autor formalizou esta
técnica e publicou uma série de trabalhos contendo diversas aplicações da mesma. A
metaheurística BT utiliza uma lista contendo o histórico da evolução do processo de busca, de
modo a evitar ciclagem; incorpora uma estratégia de balanceamento entre os movimentos
aceitos, rejeitados e aspirados; e adota procedimentos de diversificação e intensificação para o
processo de busca.\par
\cite{Souza (2000)} e \cite{White et al. (2004)} ressaltam a existência de um mecanismo, relacionado com
a Lista Tabu, que anula o status tabu de um movimento, denominado função de aspiração. Se
um movimento pode proporcionar uma melhora considerável da função objetivo, então o
status tabu é abandonado e a solução resultante é aceita como potencial vizinho.

%\subsecao{Algoritmo da colônia de formigas}	
%achar uma referencia sobre o algoritimo da colonia

\subsecao{Recozimento Simulado}

Técnica de busca local probabilística, proposta originalmente por \cite{Kirkpatrick et al. 
(1983)}, que se fundamenta em uma analogia com a termodinâmica, ao simular o 
resfriamento de um conjunto de átomos aquecidos. Isto é, conforme \cite{Noronha (2000)} em 
analogia a física da matéria: levando um cristal a sua temperatura de fusão, as moléculas 
estão desordenadas e se agitam livremente. Ao resfriar-se a amostra de maneira 
infinitamente lenta, as moléculas vão adquirir a estrutura cristalina estável que tem um 
nível de energia mais baixo possível. 
Conforme \cite{Aarts & Korst (1989)} a analogia com a otimização (combinatória ou não) 
é bastante direta. Os estados da matéria são as soluções realizáveis, a quantidade objetiva 
substitui a energia, os estados metaestáveis da matéria sendo ótimos locais e a estrutura 
cristalina corresponde ao ótimo global. 
Segundo \cite{Downsland (1993)}, a temperatura Tassume, inicialmente, um valor 
elevado T0e o procedimento pára quando a temperatura chega a um valor próximo de zero 
e nenhuma solução que piore o valor da função objetivo é mais aceita, isto é, quando o 
sistema está estável.\par
Mais informações em \cite{Downsland (1993)} e \cite{Kirkpatrick et al. (1983)}. 

\subsecao{Algoritmos genéticos}

Conforme cita \cite{Oliveira (2005)}, os algoritmos genéticos foram introduzidos 
por \cite{John Holland (Holland 1975)}, com intuito de aplicar a teoria da evolução das espécies 
elaborada por \cite{Darwin (Darwin 1859}) utilizando os conceitos da evolução biológica como 
genes, cromossomos, cruzamento, mutação e seleção na computação procurando explicar rigorosamente processos de adaptação em sistemas naturais e desenvolver sistemas artificiais (simulados em computador) que mantenham os mecanismos originais, encontrados em sistemas naturais.\par

Segundo \cite{Oliveira (2005)}, o processo de evolução executado por um algoritmo genético corresponde a um procedimento de busca no espaço de soluções potenciais para o problema e, como enfatiza \cite{Michalewicz (1992)}, esta busca requer um equilíbrio entre dois objetivos aparentemente conflitantes: a procura das melhores soluções na região que se apresenta promissora ou fase de intensificação e a procura de outra região ou exploração do espaço de busca, também conhecida como diversificação.\par
Ainda segundo \cite{Oliveira (2005)}, os algoritmos genéticos têm se mostrado ferramentas poderosas para resolver problemas onde o espaço de busca é muito grande e os métodos convencionais se mostraram ineficientes.\par

Mitchel \cite{(Mitchel 1996)} cita que a terminologia biológica é muito importante para a compreensão do funcionamento dos algoritmos genéticos. Eis os principais termos: \par
•  Cromossomo: estrutura que representa uma determinada característica da solução ou a própria solução; \par
•  Gene: característica particular de um cromossomo. O cromossomo é composto por um ou mais genes. \par
•  Alelo: valor de determinado gene;\par
•  Locus: determinada posição do gene no cromossomo;\par
•  Genótipo: estrutura que codifica uma solução. Um genótipo pode ser formado por um ou mais cromossomos; \par
•  Fenótipo: decodificação ou o significado da estrutura; \par
•  Fitness: significa aptidão. O quanto o indivíduo é apto para determinado ambiente; \par
As principais características que diferenciam os algoritmos genéticos de métodos tradicionais são \cite{(Goldberg 1989)}: \par
•  Parâmetros: os algoritmos genéticos trabalham com a codificação dos parâmetros e não com os parâmetros propriamente; \par
•  Número de soluções: os algoritmos genéticos trabalham com uma população de indivíduos (representando um conjunto de soluções) e não com uma única solução; \par

•  Avaliação das soluções: os algoritmos genéticos utilizam informações de custo ou recompensa penalizando ou premiando determinadas características das soluções; \par

•  Regras: os algoritmos genéticos utilizam regras probabilísticas e não determinísticas; \par

O algoritmo genético é uma forma da estratégia gerar-e-testar realizando os testes baseados nos parâmetros da evolução biológica. Uma desvantagem notável é a variação dos operadores genéticos do algoritmo em cada problema. Dessa forma, para resolução de determinado problema, torna-se necessário um estudo particular a respeito do mesmo. \par

O algoritmo genético atua sobre uma população fazendo com que esta evolua de acordo com uma função de avaliação. O funcionamento é iterativo iniciando com a geração de uma população inicial que pode ser aleatória ou não, seguida do processo de avaliação, seleção, cruzamento e mutação, que ocorre a cada iteração até que seja atingido algum critério de parada. Os passos gerais de um algoritmo genético são ilustrados na figura 
Figura XXXX. Cada passo pode ser realizado de várias maneiras e pode variar de problema para problema \cite{Timóteo 2002}.\par 

Figura Etapas de um Algoritmo Genético Básico 

%\secao{Trabalhos Relacionados}
%trabalho do marinho que usa tabu.
%trabalho da silvia que usa Recozimento Simulado (Simulated Annealing).
%trabalho da leonardo que usa Algoritmos Genéticos (AG).
%achar algum trabalho que utiliza o algoritimo das formigas.
%falar porque o trabalho do cara se assemelha ao meu trabalho.