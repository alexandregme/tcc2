\capitulo{METODOLOGIA}

A resolução deste trabalho\par

Algoritmo gentetico, descrever o porque da utilização do mesmo e criar a função do mesmo\par
 
A modelagem de dados do sistema utiliza a UML (Universal Modeling Language) para o desenvolvimento dos seguintes diagramas:\par

a) Diagramas de Caso de Uso.\par
b) Diagrama de Classes.\par
c) Diagrama de Seqüência.\par
d) Diagrama de Atividades.\par
e) Diagrama de Estados.\par

Para a configuração do ambiente para o desenvolvimento do sistema foram feitos os seguintes passos:\par

a) Instalação do SGBD PostgreSQL.\par
b) Instalação da linguagem de programação Java.\par
c) Instalação do Play! framework.\par
d) Instalação da IDE Eclipse.\par
e) Instalação do software Astah.\par

Para o desenvolvimento da aplicação foram executados os seguintes passos:\par

a) Criação de um novo projeto atraves do Play! framework.\par
b) Adatação do projeto para ser utilizado atraves da IDE Eclipse.\par
c) Criação das tabelas no banco de dados após a análise do sistema feita.\par
d) Configurações necessarias do projeto para conexão com o banco de dados.\par
e) Desenvolvimento das Model, Controllers e Views.\par
f) Desenvolvimento do algortimo proposto para solução da alocação de salas.\par
g) Desenvolvimento dos reslatórios.\par
h) Realização de testes.\par

A arquitetura utilizada pelo sistema MVC utiliza a comunicação via REST entre as Controllers e as Views atraves de um protocolo de comunicação no formato JSON estruturado da seguinte forma:

status : Indica o código da mensagem s = sucesso, e = erro, a = um aviso.\par
data : Apresenta todas as informações necessarias para controller.\par
message : Mensagem para a interface. Esta mensagem é exibida para o usuário.\par
length : Quantidade de dados retornados.\par

\secao{Ferramentas Utilizadas}

Este trabalho conta com a utilização de tecnologias proprias para o desenvolvimento de sistemas web, foram utilizadas as seguintes ferramentas: Para SGBD foi o utilizado PostgreSQL; No back-end foi utilizado Java e o \textit{framework Play!}; No front-end as tecnologias utilizadas foram HTML, CSS, JavaScript e \textit{framework AngularJS} e a IDE utilizada \textit{Eclipse}.\par

\subsecao{Sistema de Gerenciamento de Banco de Dados}

O SGBD escolhido foi o PostgreSQL pelo fato de ser uma ferramenta open-source e que trabalha perfeitamente com o framework escolhido Play!, uma vez que utilizado em projetos anteriores não foram apresentados conflitos entre o framework e o SGBD. A seguir pode ser notar que é uma ferramenta robusta e que tem visão no mercado internacional.\par

O PostgreSQL é um poderoso sistema gerenciador de banco de dados objeto-relacional de código aberto.  Tem mais de 15 anos de desenvolvimento ativo e uma arquitetura que comprovadamente ganhou forte reputação de confiabilidade, integridade de dados e conformidade a padrões.  Roda em todos os grandes sistemas operacionais. É totalmente compatível com ACID, tem suporte completo a chaves estrangeiras, junções (JOINs), visões, gatilhos e procedimentos armazenados (em múltiplas linguagens).  Inclui a maior parte dos tipos de dados do ISO SQL:1999, incluindo INTEGER, NUMERIC, BOOLEAN, CHAR, VARCHAR, DATE, INTERVAL, e TIMESTAMP.  Suporta também o armazenamento de objetos binários, incluindo figuras, sons ou vídeos.  Possui interfaces nativas de programação para C/C++, Java, .Net, Perl, Python, Ruby, Tcl, ODBC, entre outros, e uma excepcional documentação.\cite{postgresql}

\subsecao{Ferramentas Back-end}

Foi escolhida uma linguagem de programação Java por ser orientada a objeto. Tambem foi escolhido o \textit{Play! framework}, para que o desenvolvimento aconteça de forma mais rápida, fácil e eficiente.\par

%Java
Java foi criada pela Sun Microsystems para desenvolver inovações tecnológicas em 1992, time liderado por James Gosling. O Java utiliza do conceito de máquina virtual, onde existe, entre o sistema operacional e a aplicação, uma camada extra responsável por \"traduzir\" - mas não apenas isso - o que sua aplicação deseja fazer para as respectivas chamadas do sistema operacional, onde ela está rodando no momento. Sua aplicação roda sem nenhum envolvimento com o sistema operacional, sempre conversando apenas com a JVM - Java Virtual Machine \cite{caelum}.\par

Em 2009 a Oracle comprou a Sun, fortalecendo a marca. A Oracle sempre foi, junto com a IBM, uma das empresas que mais investiram e fizeram negócios através do uso da plataforma Java. Em 2011 surge a versão Java 7 com algumas pequenas mudanças na linguagem \cite{caelum}.\par

%-----Play! Framework

The Play! É um moderno framework MVC de alta produtividade, que utiliza Java e Scala para o desenvolvimento web, open-source , utiliza templates, hibernate e JUnit  em sua arquitetura. Existe duas versões do framework Play! 1 e Play2! este trabalho utiliza a versão 1 do framework\cite{playframework}.\par


\subsecao{Ferramentas Front-end}

As ferramentes de Front-end descritas abaixo, foram escolhidas devida a gande utilização na web grande parte dos sites contem HTML, CSS ou JavaScript em algum trexo de seu código, foi escolhido tambem o framework AngularJS para que o desenvolvimento ocorra de maneira agil e mais rapida.\par

%HTML
HTML que é defindo por (\textit{HyperText Markup Language}) ou linguagem de marcação, é uma linguagem que é utilizada no desenvolvimento de paginas web \cite{html}.\par

%css
Cascading Style Sheets (CSS) é uma tecnologia utilizada para adicionar estilos como cores, fontes, espaçamentos em documentos escritos em uma linguagem de marcação como exemplo o HTML \cite{css}.\par

%JAVASCRIPT

JavaScript é uma linguagem de script utilizada no desenvolvimento de paginas na web, atualmente é a principal linguagem para programação client-side em navegadores web. Todas as paginas de HTML modernas estão usando JavaScript para adicionar funcionalidades e para se comunicar com os webServers\cite{javascript}.\par

Angularjs é um \textit{framework JavaScript} construido e mantido pelo grupo de engenheiros do Google, ele usa o HTML como uma \textit{template engine}, tudo isso no intuito de fornecer uma solução completa para o cliente-side de sua aplicação. Além disso tem total compatibilidade com as bibliotecas javascript mais utilizadas, como jQuery. É um novo conceito para desenvolvimento de web apps client-site.\cite{angularjs}\par

\subsecao{IDE}

O Eclipse é uma IDE (\textit{integrated development environment}). Diferente de uma RAD(\textit{Rapid Application Development}), onde o objetivo é desenvolver o mais rápido possível através do arrastar-e-soltar do mouse, onde montanhas de código são gerados em background, uma IDE te auxilia no desenvolvimento, evitando se intrometer e fazer muita mágica \cite{caelum}.\par

O Eclipse é a IDE líder de mercado. Formada por um consórcio liderado pela IBM, possui seu código livre. A última versão é a 4.3, mas com qualquer versão posterior a do 3.1 você terá suporte ao Java 5, 6 e 7 \cite{caelum}.\par

Está IDE foi escolhida devido ao grande reconhecimento mundial, por sua eficiencia ao se trabalhar com a linguagem de programação Java, por ser open-source e pela existencia de varias ferramentas criadas pela comunidade, para o auxilio no desenvolvimento de softwares.\par