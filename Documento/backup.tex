%metodologia
\iniciocapitulo
Compreende a revisão da literatura ou estado da arte, metodologia e exposição da pesquisa.\par

A Revisão da Literatura compõe-se da evolução do tema e ideias de diferentes autores sobre o assunto. Deve conter citações textuais ou livres, com indicação dos autores conforme norma NBR 10520/2002.\par

O Estado da Arte segundo definição no dicionário Aurélio é o nível de desenvolvimento atingido (por uma ciência, uma técnica) na atualidade, ou seja, refere-se ao quadro atual de uma área, suas tendências, potencialidades e excelência no assunto.\par

A Metodologia deve apresentar o método adotado na entrevista, questionário, observação, experimentação na população pesquisada na características e quantificação.\par

A Exposição da Pesquisa é a análise dos fatos apresentados, ou seja, os dados obtidos, as estatísticas, comparações com outros estudos e outras observações.\par

\secao{Seção}

Exemplo de citação \cite{knuth68}.

Exemplo de citação \cite{knuth69}.

Exemplo de citação \cite{knuth73}.

\subsecao{Subseção}

\subsubsecao{Subsubseção}


% Exemplo de tabela:
\begin{table}[!htb]
	\caption[Descrição na Lista de Tabelas]{Legenda da tabela}
	\centering
	\label{tab:tabela}
	\begin{tabular}{c|c|c}
		\hline
		Coluna 1 & Coluna 2 & Coluna 3 \\
		\hline
 		Célula 1 & Célula 2 & Célula 3 \\
		Célula 4 & Célula 5 & Célula 6 \\
		Célula 7 & Célula 8 & Célula 9 \\
		\hline
	\end{tabular}
	\\ \textbf{\footnotesize Fonte: Fonte da tabela}
\end{table}

% Exemplo de figura:

\begin{figure}[!htb]
   \caption[Descrição na Lista de Figuras]{Legenda da figura}
   \label{fig:figura1}
   \centering
   \includegraphics{LogoPUC.jpg}
   \\ \textbf{\footnotesize Fonte: Fonte da figura}
\end{figure}