% Nome do capitulo
\chapter{Metodologia}
% Label para referenciar
\label{cap:metodologia}
% Diminuir espacamento entre titulo e texto
\vspace{-1.9cm}

% Texto do capitulo
Primeiramente para esse projeto foram realizadas as pesquisas bibliogr�ficas. Trabalhos de outros autores que de alguma forma se relacionam com a monografia foram estudados com a finalidade fornecer uma base de conhecimento sobre o tema, como por exemplo,  o entendimento de todo o processo de descoberta do conhecimento e suas t�cnicas de minera��o de dados. Com isso foi poss�vel montar os cap�tulos de revis�o bibliogr�ficas abordando os principais conceitos utilizados durante o desenvolvimento.

No segundo momento, o desenvolvimento trata-se da aplica��o dos passos do processo de KDD sobre os Microdados do Censo da Educa��o Superior. Foram executados os seguintes passos: sele��o de dados, pr�-processamento, transforma��o, minera��o de dados e interpreta��o.Na sele��o de dados foi definida a fonte dos dados. No pr�-processamento removemos principalmente erros relacionados com a falta de informa��o. Na transforma��o os dados foram enriquecidos. Partindo para a minera��o de dados, os algoritmos de influencias, previs�o, clusteriza��o, e an�lise de cesta de compras foram aplicados. Finalmente, na interpreta��o dos dados os gr�ficos e an�lises encontradas s�o apresentados.