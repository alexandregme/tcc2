% --- Campos que formarao: capa, folha de rosto, folha de aprovacao  ---

\curso{Bacharelado em Sistemas de Informação}

\autor{Alexandre Gonzaga Mendes}

\titulo{DEFINIR O NOME} % Caixa alta

\subtitulo{Subtítulo do Trabalho} % Caixa baixa

\cidade{Belo Horizonte}

\ano{2013}

\trabalho{Monografia} % Tipo de trabalho: Monografia, Dissertacao, Tese...

\grau{Bacharel em Sistemas de Informação} % Grau do trabalho: Bacharel em..., Mestre em..., Doutor em...

\orientador{João Caram Adriano}

\avaliadorA{Nome do Avaliador 1}

\avaliadorB{Nome do Avaliador 2}

\datacompleta{DD de MM de 2013} % Dia, mes e ano

% --- CAPA ---

\capa

% --- FOLHA DE ROSTO ---

\folharosto

% --- FOLHA DE APROVACAO ---

\folhaaprovacao

% --- DEDICATORIA (elemento opcional) ---

\dedicatoria
{
\textit{Dedicatória: Página onde o autor presta homenagem a uma ou mais pessoas.
O layout desta página fica a critério do autor, mas o tipo e tamanho de letras são definidos pela ABNT.}
}

% --- AGRADECIMENTOS (elemento opcional) ---

\agradecimentos
{
Agradecimentos a pessoas que contribuíram para o desenvolvimento do trabalho.
Agradecimentos a pessoas que contribuíram para o desenvolvimento do trabalho.
}

% --- EPIGRAFE (elemento opcional) ---

\epigrafe
{
\textit{Epígrafe: Pensamentos retirados de um livro, uma música, um poema, normalmente relacionados ao tema do trabalho.
Deve ser elaborada conforme norma NBR 10520/2002. Apresentação de citações em documentos.
Se desejar, a epígrafe pode ser grafada em itálico.
Ao final do trabalho deve-se fazer a referência completa da publicação de onde a epígrafe foi retirada.}
}

% --- RESUMO ---

\resumo
{
Apresentação concisa dos pontos relevantes do texto. Deve ressaltar o objetivo, o método, resultados e conclusões do trabalho. Deve-se utilizar o verbo na voz ativa ou terceira pessoa do singular. O resumo não deve conter citações ou indicações bibliográficas.
} % Resumo
{
Ao final do resumo deve-se elaborar palavras-chave representativas do conteúdo do trabalho, separadas entre si por um ponto.
} % Palavras-chave

% --- ABSTRACT ---

\abstract
{
Versão do resumo em idioma de divulgação internacional. Deve ser a tradução literal do resumo em português e apresentar palavras- chave no mesmo idioma, logo abaixo do texto, separadas entre si por um ponto.
} % Abstract
{
O resumo em língua estrangeira também deve conter palavras-chave representativas do conteúdo do trabalho, separadas entre si por um ponto.
} % Keywords

% --- LISTA DE FIGURAS, LISTA DE TABELAS, SUMARIO ---

\listafiguras
\listatabelas
\listasiglas {
\sigla{S1}{Sigla 1}
\sigla{S2}{Sigla 2}
\sigla{S3}{Sigla 3}
}


% --- TEXTO ---