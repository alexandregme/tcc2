% Resumo
\begin{resumo}

% Texto do resumo, em portugu�s: sem par�grafo, alinhado � esquerda
\noindent 

O \ac{KDD} � um processo composto de v�rias etapas para compreens�o de padr�es nos dados. Dada a divulga��o p�blica dos dados do Censo da Educa��o Superior realizada anualmente pelo \ac{Inep} temos uma base de dados para desenvolver o processo. Foi utilizada Minera��o de Dados, com o aux�lio de ferramentas como o SQL Server e Excel para descoberta de conhecimento nessa base de dados. Visto que um dos maiores desafios que o ensino superior enfrenta hoje � prever as decis�es dos alunos, a utiliza��o desse processo e ferramentas pode ajudar a tomada de decis�es da Universidade PUC Minas. Os resultados trouxeram informa��es e previs�es sobre ingressos e evas�es; an�lises sobre a quantidade de candidatos vaga; a import�ncia do curso de Sistemas de Informa��o dentro e fora da PUC Minas; influenciadores da taxa de ocupa��o, principais cursos que aparecem juntos com grande ocupa��o e recomenda��es.

% Espa�amento para as palavras-chave
\vspace*{.75cm}

% Palavras-chave: sem par�grafo, alinhado � esquerda
\noindent Palavras-chave: Processo KDD. SQL Server. Excel. ETL. Minera��o de Dados.  \\
% Segunda linha de palavras-chave, com espa�amento.
\indent\hspace{2cm}Censo da Educa��o Superior.

\end{resumo}
