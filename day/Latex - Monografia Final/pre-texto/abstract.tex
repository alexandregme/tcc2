% Abstract
\begin{abstract}

% Texto do resumo, em ingl�s: sem par�grafo, alinhado � esquerda
\noindent \ac{KDD} is one process that consists of several steps to understanding patterns in data. Given the public disclosure of data from the Census of Higher Education conducted annually by Inep we have a great database to develop the process. Data Mining was used, with the aid of tools such as SQL Server and Excel for knowledge discovery in this database. Since one of the biggest challenges that higher education faces today is predicting the decisions of students, the use of this process and tools can help decision making at the University PUC Minas.

% Espa�amento para as palavras-chave
\vspace*{.75cm}

% Palavras-chave: sem par�grafo, alinhado � esquerda
\noindent Keywords: KDD Process. SQL Server. Excel. ETL. Data Mining. \\
% Segunda linha de palavras-chave, com espa�amento.
\indent\hspace{1.4cm}Census of Higher Education.

\end{abstract}
